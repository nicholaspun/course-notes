\documentclass[12pt]{article}

\usepackage[utf8]{inputenc}
\usepackage[english]{babel}
\usepackage{amsmath, amssymb, amsthm}
\usepackage{fullpage}
\usepackage{hyperref}
\usepackage{parskip}

\newcommand{\R}{\mathbb{R}}
\newcommand{\N}{\mathbb{N}}
\newcommand{\Q}{\mathbb{Q}}
\newcommand{\Z}{\mathbb{Z}}

\theoremstyle{definition}
\newtheorem{definition}{Definition}[section]
\theoremstyle{remark}
\newtheorem*{remark}{Remark}
\newtheorem{theorem}{Theorem}[section]
\newtheorem{corollary}{Corollary}[theorem]
\newtheorem{lemma}[theorem]{Lemma}

\setcounter{section}{-1}

\numberwithin{equation}{section}

\begin{document}
\begin{center}
    \Large CO450 - Combinatorial Optimization (Fall 2019)
\end{center}

\tableofcontents

\section{Introduction}
\subsection{Overview of the Course}
Combinatorial optimization leverages tools from: combinatorics, linear programming theory and algorithms to \textit{efficiently} solve optimization problems on discrete structures (e.g. graphs)

The course will covering the following topics:
\begin{itemize}
    \item Spanning frees - \textit{Given connected, undirected graphs with edge costs, find the minimum spanning tree}
    \item Max flow, Min cut
    \item Matroids and matroid optimization
    \item Matchings and related problems
    \item Approximation algorithms
\end{itemize}

\subsection{Review of LP theory}
A \underline{linear program} (LP) is an optimization problem of the form:
\begin{equation}\label{eq:intro-review-lp}
\begin{aligned}
    \max \quad c^\intercal x & \\
    \text{s.t.} \quad Ax &\leq b \\
    x &\geq 0
\end{aligned}
\end{equation}
where $x \in \R^n, \ A \in M_{m \times n}(\R) $, and the objective function and constraints are linear. We must also require that:
\begin{itemize}
    \item There are a finite number of variables and constraints
    \item The inequalities are non-strict
\end{itemize}

Any LP has 3 possible outcomes:
\begin{enumerate}
    \item The LP is \underline{infeasible}
    \item The LP is \underline{unbounded}, i.e. We can achieve feasible solutions of arbitrarily ``good" objective value. (For \eqref{eq:intro-review-lp}, this means that $\forall v \in \R$ there exists a feasible solution $x$ s.t $c^\intercal x > v$) 
    \item The LP has an \underline{optimal solution}. (For \eqref{eq:intro-review-lp}, this means there is a feasible solution $x^*$ such that $c^\intercal x^* \geq c^\intercal x \ \forall \ \text{feasible solutions} \ x$)
\end{enumerate}


\end{document}