\section{Introduction}

This course is concerned with combinatorial objects satisfying certain regularity/balanced conditions.
We will go over the following topics:
\begin{itemize}
    \item Classical results in design theory
    \item Constructions of designs
    \item Structural theorems
    \item Non-existence results
    \item Lots of algebra (abstract and linear)
    \item Some number theory!
\end{itemize}

\subsection{The 36 Officers Problem (Euler, 1782)}
\ul{Problem statement}: We have 36 pieces that are 
\begin{itemize}
    \item 1 of 6 chess pieces: King, Queen, Bishop, Rook, Knight, Pawn, and
    \item 1 of 6 colors (one piece of each colour)
\end{itemize}

Can we arrange these on a 6-by-6 square such that each row and each column uses exactly one piece of each type and one piece of each colour?
Euler thought this was impossible.

Note that this is easy for 5 chess pieces and 5 colours.
Consider (using numbers 1 through 5 in place of the different chess pieces):
\begin{center}
    \begin{tabular}{|c|c|c|c|c|}
        \hline
        \textcolor{cyan}{1} & \textcolor{red}{2} & \textcolor{yellow}{3} & \textcolor{green}{4} & \textcolor{pink}{5} \\
        \hline
        \textcolor{pink}{2} & \textcolor{cyan}{3} & \textcolor{red}{4} & \textcolor{yellow}{5} & \textcolor{green}{1} \\ 
        \hline
        \textcolor{green}{3} & \textcolor{pink}{4} & \textcolor{cyan}{5} & \textcolor{red}{1} & \textcolor{yellow}{2} \\ 
        \hline
        \textcolor{yellow}{4} & \textcolor{green}{5} & \textcolor{pink}{1} & \textcolor{cyan}{2} & \textcolor{red}{3} \\ 
        \hline
        \textcolor{red}{5} & \textcolor{yellow}{1} & \textcolor{green}{2} & \textcolor{pink}{3} & \textcolor{cyan}{4} \\ 
        \hline
    \end{tabular}
\end{center}
Which was constructed by taking the sequence 12345 and shifting it left each row.

Further, this is impossible for 2 pieces and 2 colours.
In fact, Euler conjectured that this is impossible for $n \equiv 2 \pmod{4}$

However, this was wrong! (In fact, 2 ad 6 are the only numbers for which this is impossible, and we'll gain a better understanding of why through this course!)

\subsection{Statistical Experimental Design}
Here is a take on design theory.
Consider the following scenario where:
\begin{itemize}
    \item We have \ul{$v$ kinds} of wine we want to compare,
    \item But, we can only accurately compare \ul{$k$ kinds} per day.
    \item And, we only want to compare each pair \ul{exactly once}.
\end{itemize}
Can we decide on what to drink each day in order to accomplish these tasks?

\begin{example}[Fano Plane]\label{lec1:exmp-fano-plane}
    Suppose $v = 7, k = 3$.

    We will call the objects we want to compare (in this case, the wines) \ul{points} and the experiment for the day a \ul{block}.

    Here our points be labelled with numbers: $\{0, 1, 2, \ldots, 6\}$

    And our blocks for each day will be:
    \begin{enumerate}[label=]
        \item Day 1: $\{0, 1, 3\}$
        \item Day 2: $\{1, 2, 4\}$
        \item Day 3: $\{2, 3, 5\}$
        \item Day 4: $\{3, 4, 6\}$
        \item Day 5: $\{4, 5, 0\}$
        \item Day 6: $\{5, 6, 1\}$
        \item Day 7: $\{6, 0, 2\}$
    \end{enumerate}
    Note that we can obtain the next day's experiment by adding 1 modulo 7.
    And, finally we can check that every pair appears in exactly one block.
\end{example}
This example actually has a nice visualization:

\begin{minipage}{\textwidth}
    \centering
    \begin{tikzpicture}[scale = 0.7]
        \draw (0:0) circle (2);

        \node[vertex] (0) at (210:4) {0};
        \node[vertex, fill=white] (1) at (150:2) {1};
        \node[vertex, fill=white] (2) at (30:2) {2};
        \node[vertex] (3) at (90:4) {3};
        \node[vertex, fill=white] (4) at (270:2) {4};
        \node[vertex] (5) at (330:4) {5};
        \node[vertex] (6) at (0:0) {6};

        \draw (0) -- (1) -- (3) -- (2) -- (5) -- (4) -- (0) -- cycle;
        \draw   (0) -- (6)
                (1) -- (6)
                (2) -- (6)
                (3) -- (6)
                (4) -- (6)
                (5) -- (6);
    \end{tikzpicture}
\end{minipage}

Note that each collinear line gives us a block (We count the line through 1, 2, and 4 as a ``straight'' line)
How did we come up with this?
We'll answer this through the course!
For now, let's look at another example:

\begin{example}[Game of Set]
    In this game, we have 81 cards, each with 4 attributes:
    \begin{itemize}
        \item Colour: Red, Green, or Purple
        \item Number: 1, 2, or 3
        \item Shading: Open, Shaded, or Solid
        \item Shape: Oval, Diamond, or Squiggle
    \end{itemize}
    
    A \ul{Set} is a triple of cards with the property that in each attribute, they are all the same or all different.

    \begin{note}
        Each pair of cards is in a unique Set!
    \end{note}
    This is since on each attribute, the pair is either matching or different, and there is a unique third card that can complete the pattern.
\end{example}

How do the above examples compare?

\begin{minipage}{\textwidth}
    \captionof*{table}{\textbf{Balancing Properties of the Two Examples}}
    \begin{center}        
    \begin{tabular}{c|c}
        Fano Plane & Set \\
        \hline
        Each block has 3 points & Each Set has 3 cards \\
        Each point is in 3 blocks & Each card is in 40 different Sets* \\
        Each pair of points is in exactly one block & Each pair of cards is in a unique set
    \end{tabular}
    \end{center}
\end{minipage}

*Why is this true? For every card, we can choose any other card in the deck (There are 80 such choices)
But, we observed earlier that for every pair, there is a unique 3rd card that completes a Set.
So, in fact, we'll treat the remaining 80 cards as pairs.

\subsection{Basic Definitions}
\begin{definition}
    A \ul{design} is a pair $(V, \B)$ where
    \begin{itemize}
        \item $V$ is a finite set of elements called \ul{points}
        \item $\B$ is a collection (multiset) of (usually non-empty) subsets of $V$, called \ul{blocks}
    \end{itemize}
\end{definition}

We call a design \ul{simple} if $\B$ is a set (there are no repeated blocks)

\begin{definition}
    A \ul{$t$-design} ($t \in \N$) is a design $(V, \B)$ in which all blocks have the same size $k$, and there is a constant $\lambda_t$ such that every $t$-tuple of points lie in exactly $\lambda_t$ blocks.
\end{definition}

The two examples we just saw were $2$-designs with $\lambda_2 = 1$ and also $1$-designs with $\lambda_1 = 3$ for the Fano plane and $\lambda_1 = 40$ for the game of Set.

\begin{definition}\label{lec1:defn-bibd}
    A \ul{Balanced Incomplete Block Design} (BIBD) is a design that is a $2$-design and a $1$-design.
\end{definition}

\begin{exercise}
    If $\lambda_2 \neq 0$, then show that every 2-design is automatically a 1-design
\end{exercise}