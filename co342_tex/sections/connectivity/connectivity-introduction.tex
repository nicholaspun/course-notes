\section{Introduction}

\begin{note}
    In this course, we will assume a graph $G = (V, E)$ is simple, unless otherwise stated.
\end{note}

We are interested answering two questions in this chapter:
\begin{itemize}
    \item How many vertices/edges need to be removed in order to disconnect a graph?
    \item Assuming we know that number, can we characterize that class of graphs?
\end{itemize}
The first question will take us to \Cref{sec:connectivity-mengers} and the second will span the latter half of this chapter.

We start of with two basic definitions, the first of which the reader should already be familiar with:

\begin{definition}{Induced Subgraph}{induced-subgraph}
    Let $S$ be a set of vertices in $G$. Then, the \bf{subgraph induced} by $S$, denoted $G[S]$ consists of $S$ as the vertex set and all edges in $G$ joining 2 vertices in $S$.
\end{definition}


\begin{example}
    Let $S = \{1, 2, 3, 4\}$ and $T = \{1, 2, 4, 5\}$, then $G[S]$ and $G[T]$ are as pictured in \Cref{fig:connectivity-introduction-subgraph-examples}
\end{example}

\begin{example}[Induced Cycle]
    Let $U = \{1, 2, 3\}$, then $G[U]$ is called an \ul{induced cycle}. 
\end{example}

\begin{figure}[htbp]
  \centering
  \begin{subfigure}{.24\textwidth}
    \centering
    \begin{tikzpicture}[scale = 0.8, transform shape]
        \node[vertex] (1) at (0,0) {1};
        \node[vertex, below right=of 1] (2) {2};
        \node[vertex, below left=of 1] (5) {5};
        \node[vertex, below right=of 5, below left=of 2] (3) {3};
        \node[vertex, below right=of 3] (4) {4};
        \node[vertex, below left=of 3] (6) {6};
        
        \draw[edge] (1) -- (2) -- (3) -- (5) -- (1);
        \draw[edge] (1) -- (3) -- (4);
        \draw[edge] (3) -- (6);

        \draw[edge] (4) to node[below, yshift=-0.25cm]{$G$} (6);
    \end{tikzpicture}
  \end{subfigure}
  \begin{subfigure}{.24\textwidth}
    \centering
    \begin{tikzpicture}[scale = 0.8, transform shape]
        \node[vertex] (1) at (0,0) {1};
        \node[vertex, below right=of 1] (2) {2};
        \node[below left=of 1] (5) {};
        \node[vertex, below right=of 5, below left=of 2] (3) {3};
        \node[vertex, below right=of 3] (4) {4};
        \node[below left=of 3] (6) {};
        
        \draw[edge] (1) -- (2) -- (3) -- (1);
        \draw[edge] (3) -- (4);

        \path (4) to node[below, yshift=-0.5cm]{$G[S]$} (6);
    \end{tikzpicture}
  \end{subfigure}
  \begin{subfigure}{.24\textwidth}
    \centering
    \begin{tikzpicture}[scale = 0.8, transform shape]
        \node[vertex] (1) at (0,0) {1};
        \node[vertex, below right=of 1] (2) {2};
        \node[vertex, below left=of 1] (5) {5};
        \node[below right=of 5, below left=of 2] (3) {};
        \node[vertex, below right=of 3] (4) {4};
        \node[below left=of 3] (6) {};
        
        \draw[edge] (5) -- (1) -- (2);

        \path (4) to node[below, yshift=-0.5cm]{$G[T]$} (6);
    \end{tikzpicture}
  \end{subfigure}
  \begin{subfigure}{.24\textwidth}
    \centering
    \begin{tikzpicture}[scale = 0.8, transform shape]
        \node[vertex] (1) at (0,0) {1};
        \node[vertex, below right=of 1] (2) {2};
        \node[below left=of 1] (5) {};
        \node[vertex, below right=of 5, below left=of 2] (3) {3};
        \node[below right=of 3] (4) {};
        \node[below left=of 3] (6) {};
        
        \draw[edge] (1) -- (2) -- (3) -- (1);

        \path (4) to node[below, yshift=-0.5cm]{$G[U]$} (6);        
    \end{tikzpicture}
  \end{subfigure}

  \caption{}
  \label{fig:connectivity-introduction-subgraph-examples}
\end{figure}

\begin{definition}{Min/Max Degree}{min-and-max-degree}
    We use $\delta(G)$ to denote the minimum degree among all vertices of $G$ and $\Delta(G)$ for the maximum degree
\end{definition}

\begin{example}
    With $G$ in \Cref{fig:connectivity-introduction-subgraph-examples}, we have $\delta(G) = 2$ and $\Delta(G) = 5$.
\end{example}

\subsection{Edge Connectivity}
We start our investigation into connectivity with removing edges from graphs.
Recall the following definition from MATH 239:
\begin{definition}{Bridge/Cut-edge}{bridge}
    A \bf{bridge} or \bf{cut-edge} in $G$ is an edge whose removal increases the number of components in $G$.
\end{definition}

Let's extend this definition to more general sets:
\begin{definition}{Disconnecting Set}{disconnecting-se }
    A set of edges $F$ is called a \bf{disconnecting set} if removing all edges in $F$ from $G$ results in a disconnected graph. We use $G - F$ to denote this removal
\end{definition}

A couple remarks on this definition:
\begin{itemize}
    \item If a graph is already disconnected, the empty set is a disconnecting set
    \item In general, we're interested in minimizing the size of a disconnecting set. (Since removing every edge trivially creates a disconnected graph)
\end{itemize}

\newpage
\begin{example}\label{exmp:connectivity-introduction-disconnecting-set}
    \ \\
    \begin{minipage}{.78\textwidth}
        Concerning the graph on the right, some disconnecting sets are $\{12, 13\}$, $\{12, 23, 34, 36\}$ and the set of all edges. Note that since this graph does not have a bridge, we must remove at least 2 edges in order to disconnect the graph.
    \end{minipage}\hfill
    \begin{minipage}{.19\textwidth}
        \centering
        \begin{tikzpicture}[scale=0.75, transform shape]
            \node[vertex] (1) at (0,0) {1};
            \node[vertex, below left=of 1] (2) {2};
            \node[vertex, below right=of 1] (3) {3};
            \node[vertex, below right=of 2, below left=of 3] (4) {4};
            \node[vertex, below left=of 4] (5) {5};
            \node[vertex, below right=of 4] (6) {6};
            
            \draw[edge] (1) -- (2) -- (3) -- (1);
            \draw[edge] (3) -- (6) -- (4) -- (5) -- (2);
            \draw[edge] (2) -- (4) -- (3);
            \draw[edge] (5) -- (6);
        \end{tikzpicture}
    \end{minipage}
\end{example}

When characterizing graphs by the size of their disconnecting sets, we'll use the following two terms:
\begin{definition}{$k$-edge-connected}{k-edge-connected}
    A graph is \bf{$k$-edge-connected} if every disconnecting set has size $\geq k$.
\end{definition}
\begin{definition}{Edge Connectivity}{edge-connectivity}
    The \bf{edge connectivity} of $G$, written $\kappa'(G)$ is the largest $k$ for which $G$ is $k$-edge-connected, or equivalently, the minimum size of a disconnecting set.
\end{definition}

\begin{example}
    \leavevmode
    \begin{itemize}
        \item A connected graph is $1$-edge-connected and if there is a bridge, then the edge connectivity is $1$. (The bridge is the minimum disconnecting set)
        \item A connected graph with \bf{no} bridges is $2$-edge-connected
        \item A graph with a single vertex has edge connectivity $0$ (The only disconnecting set is the empty set)
    \end{itemize}
\end{example}

\begin{example}
    The graph from \Cref{exmp:connectivity-introduction-disconnecting-set} is $2$-edge-connected (since it has no bridge), but \bf{not} $3$-edge-connected (since we've shown a disconnecting set of size $2$).
    So $\kappa'(G) = 2$
\end{example}

\begin{example}
    If a graph is $k$-edge-connected, then it is also $(k-1)$-edge-connected
\end{example}

We can always disconnect any vertex of a graph by removing all of its incident edges.
Since we want to minimize the size of the disconnecting set, we can choose a vertex of minimum degree.
This gives the following bound:
\begin{proposition}
    {}
    {connectivity-introduction-edge-connectivity-bounded-by-min-degree}
    $\kappa'(G) \leq \delta(G)$
\end{proposition}
\begin{proof}
    Assume $G$ is non-trivial and let $v$ be a vertex of degree $\delta(G)$.
    Let $F$ be the set of all edges incident with $v$, then $G - F$ has no path from $v$ to any other vertex.
    So, $F$ is a disconnecting set of size $\delta(G)$, so $G$ is not $(\delta(G) + 1)$-edge-connected and so $\kappa'(G) \leq \delta(G)$
\end{proof}



\subsection{Vertex Connectivity}