\subsection{Ford's Algorithm}
In this section, we will create an algorithm that will find all shortest $s,v$-dipaths (provided there are no negative dicycles).
Note that it will not be able to detect negative dicycles.
Although this may not be ideal, we will find that this algorithm serves as a good starting point for the discussion of the algorithms that come after.

\underline{Some motivation for the algorithm:}

Consider the following path:

\begin{minipage}{\textwidth}
    \centering
    \begin{tikzpicture}
        \node[vertex] (s) at (0,0) {$s$};
        \node[vertex] (v1) at (2,0) {$v_1$};
        \node[vertex] (v2) at (4,0) {$v_2$};
        \node[vertex] (v3) at (6,0) {$v_3$};
        \node[vertex] (t) at (8,0) {$t$};

        \begin{scope}[every node/.style={fill = white, ellipse}]
            \draw[edge] (s) edge (v1);
            \draw[edge] (v1) edge (v2);
            \draw[edge] (v2) edge (v3);
            \draw[edge] (v3) edge (t);
        \end{scope}
    \end{tikzpicture}
\end{minipage}

If there are no negative dicycles, then this shortest $s,t$-dipath also contains the shortest $s,v_3$-dipath, $s,v_2$-dipath, and $s,v_1$-dipath.

It is worth seeing an example where there are negative dicycles:

\begin{minipage}{\textwidth}
    \centering
    \begin{tikzpicture}
        \node[vertex] (s) at (0,0) {$s$};
        \node[vertex] (v1) at (3,0) {$v_1$};
        \node[vertex] (v2) at (6,0) {$v_2$};
        \node[vertex] (t) at (9,0) {$t$};

        \begin{scope}[every node/.style={fill = white, ellipse}]
            \draw[edge] (s) to node {$1$} (v1);
            \draw[edge, bend left] (s) to node {$4$} (t);
            \draw[edge] (v1) to node {$1$} (v2);
            \draw[edge] (v2) to node {$1$} (t);
            \draw[edge, bend left] (t) to node {$-4$} (v1);
        \end{scope}
    \end{tikzpicture}
\end{minipage}

Here, rather than taking the arc $su$ for a cost of $1$, the cheaper path is $stu$ for a cost of $0$.

So, in the absence of negative dicycles, let's prove our motivating idea:
\begin{theorem}{}{}
    Let $D = (N, A)$ be a digraph with arc costs $w \in \R^A$ with no negative dicycles. 
    If $v_1v_2\ldots v_k$ is the shortest $v_1,v_k$-dipath, then $v_1\ldots v_i$ is the shortest $v_1,v_i$-dipath.
\end{theorem}
\begin{proof}
    Since there are no negative dicycles, our LP has an optimal solution which is the characteristic vector of an $v_1,v_k$-dipath $P$.
    Then, there is a corresponding optimal dual solution $y$ where all arcs of $P$ are equality arcs (due to CS conditions).
    Let $P': v_1\ldots v_i$ for some $i<k$.
    $y$ is still feasible for the dual LP of the shortest $v_1,v_i$-dipath problem and any arc in $P'$ is in $P$, so all arcs of $P'$ are equality arcs.
    So, the CS conditions are satisfied for the $v_1,v_i$-dipath problem
\end{proof}
\begin{definition}{Rooted Tree (at $s$)}{}
    A tree $T$ is \underline{rooted} at $s$ if for all $t \in N(T)$, the unique $s,t$-path in $T$ is an $s,t$-dipath
\end{definition}
We can build a rooted tree at $s$ representing the shortest $s,v$-dipaths for all $v \in N\setminus\{s\}$ by solving the single $s,v$-dipath problem multiple times.
However, we can, in fact, do much better than this, and we'll show that we can find them all at once.

First, we'll need two results on rooted trees:
\begin{proposition}{}{}
    There is an $s,t$-dipath in $D$ for all $t \in N$ if and only if there exists a spanning tree in $D$ rooted at $s$
\end{proposition}
\begin{proposition}{}{}
    Let $T$ be a spanning tree in $D$. Then, $T$ is rooted at $s$ if and only if the in-degree of $s$ is $0$, and the in-degree of all other nodes is $1$ in $T$
\end{proposition}
\begin{proof}
    TODO!
\end{proof}

\begin{definition}{Predecessor}{}
    In a spanning tree $T$, rooted at $s$, for each node $v \in N\setminus\{s\}$, its \underline{predecessor} is the unique node $u$ so that $uv$ is an arc in $T$. We write this as $p_v = u$
\end{definition}

\subsubsection{The Algorithm}
At each step of the algorithm, we maintain a potential and the predecessor of each node.

\IncMargin{1em}
\begin{algorithm}[H]\label{alg:ford}  
  \nl \textbf{Initialization:}
  \begin{itemize}
      \item Set $y_s = 0$ and $y_v = \infty\:\forall v \in N\setminus\{s\}$ 
      \item Set $p_v =$ undefined $\forall v \in N$
  \end{itemize}

  \nl \While{$y$ is \underline{not} feasible}{
    \begin{enumerate}[label=(\alph*)]
        \item Find an arc $uv \in A$ where $\overline{w}_{uv} < 0$ (So $y_u + w_{uv} < y_v$)
        \item Set $y_v = y_u + w_{uv}$ (i.e. Make it an equality arc)
        \item Set $p_v = u$
    \end{enumerate}
  }

  \caption{Ford's Algorithm}
\end{algorithm}

\begin{example}{Ford's Algorithm}{}
    We'll run Ford's algorithm to find the spanning tree rooted at $s$ in the digraph below. 

    \begin{minipage}{\textwidth}
        \centering
        \begin{tikzpicture}[scale = 0.75]
            \node[vertex] (s) at (0,0) {$s$};
            \node[vertex] (a) at (4,2) {$a$};
            \node[vertex] (b) at (4,-2) {$b$};
            \node[vertex] (c) at (8,0) {$c$};
    
            \begin{scope}[every node/.style={fill = white, ellipse}]
                \draw[edge] (s) to node {$-1$} (a);
                \draw[edge] (s) to node {$2$} (b);
                \draw[edge] (a) to node {$5$} (c);
                \draw[edge] (b) to node {$-4$} (a);
                \draw[edge] (b) to node {$2$} (c);
            \end{scope}
        \end{tikzpicture}
    \end{minipage}

    \underline{Initialization:} We've labelled the nodes with their dual potentials. (Note: UD stands for undefined)
    
    \begin{minipage}{\textwidth}
        \centering
        \begin{minipage}{0.6\textwidth}
            \centering
            \begin{tikzpicture}[scale = 0.6]
                \node[vertex][label = left:{$0$}] (s) at (0,0) {$s$};
                \node[vertex][label = above:{$\infty$}] (a) at (4,2) {$a$};
                \node[vertex][label = below:{$\infty$}] (b) at (4,-2) {$b$};
                \node[vertex][label = right:{$\infty$}] (c) at (8,0) {$c$};
        
                \draw[edge] (s) edge (a);
                \draw[edge] (s) edge (b);
                \draw[edge] (a) edge (c);
                \draw[edge] (b) edge (a);
                \draw[edge] (b) edge (c);
            \end{tikzpicture}
        \end{minipage}%
        \begin{minipage}{0.4\textwidth}
            \centering
            \begin{tabular}{c|ccc}
                Nodes & $a$ & $b$ & $c$ \\
                \hline
                Preds. & UD & UD & UD             
            \end{tabular}
        \end{minipage}
    \end{minipage}

    \underline{Iteration 1:} $\overline{w}_{sa} = 1 - y_a = -\infty < 0$, so we correct $sa$ by setting $y_a = -1$. The predecessor for $a$ is set to be $s$
    
    \begin{minipage}{\textwidth}
        \centering
        \begin{minipage}{0.6\textwidth}
            \centering
            \begin{tikzpicture}[scale = 0.6]
                \node[vertex][label = left:{$0$}] (s) at (0,0) {$s$};
                \node[vertex][label = above:{$\infty$}] (a) at (4,2) {$a$};
                \node[vertex][label = below:{$\infty$}] (b) at (4,-2) {$b$};
                \node[vertex][label = right:{$\infty$}] (c) at (8,0) {$c$};
        
                \draw[edge, red] (s) edge (a);
                \draw[edge] (s) edge (b);
                \draw[edge] (a) edge (c);
                \draw[edge] (b) edge (a);
                \draw[edge] (b) edge (c);
            \end{tikzpicture}
        \end{minipage}%
        \begin{minipage}{0.4\textwidth}
            \centering
            \begin{tabular}{c|ccc}
                Nodes & $a$ & $b$ & $c$ \\
                \hline
                Preds. & UD & UD & UD             
            \end{tabular}
        \end{minipage}
    \end{minipage}

    \underline{Iteration 2:} $\overline{w}_{ac} = -1 + 5 - y_c = -\infty < 0$, so we correct $ac$ by setting $y_c = 4$. The predecessor for $c$ is set to be $a$
    
    \begin{minipage}{\textwidth}
        \centering
        \begin{minipage}{0.6\textwidth}
            \centering
            \begin{tikzpicture}[scale = 0.6]
                \node[vertex][label = left:{$0$}] (s) at (0,0) {$s$};
                \node[vertex][label = above:{$-1$}] (a) at (4,2) {$a$};
                \node[vertex][label = below:{$\infty$}] (b) at (4,-2) {$b$};
                \node[vertex][label = right:{$\infty$}] (c) at (8,0) {$c$};
        
                \draw[edge] (s) edge (a);
                \draw[edge] (s) edge (b);
                \draw[edge, red] (a) edge (c);
                \draw[edge] (b) edge (a);
                \draw[edge] (b) edge (c);
            \end{tikzpicture}
        \end{minipage}%
        \begin{minipage}{0.4\textwidth}
            \centering
            \begin{tabular}{c|ccc}
                Nodes & $a$ & $b$ & $c$ \\
                \hline
                Preds. & $s$ & UD & UD             
            \end{tabular}
        \end{minipage}
    \end{minipage}

    \underline{Iteration 3:} $\overline{w}_{sb} = 2 - \infty = -\infty < 0$, so we correct $sb$ by setting $y_b = 2$. The predecessor for $b$ is set to be $s$
    
    \begin{minipage}{\textwidth}
        \centering
        \begin{minipage}{0.6\textwidth}
            \centering
            \begin{tikzpicture}[scale = 0.6]
                \node[vertex][label = left:{$0$}] (s) at (0,0) {$s$};
                \node[vertex][label = above:{$-1$}] (a) at (4,2) {$a$};
                \node[vertex][label = below:{$\infty$}] (b) at (4,-2) {$b$};
                \node[vertex][label = right:{$4$}] (c) at (8,0) {$c$};
        
                \draw[edge] (s) edge (a);
                \draw[edge, red] (s) edge (b);
                \draw[edge] (a) edge (c);
                \draw[edge] (b) edge (a);
                \draw[edge] (b) edge (c);
            \end{tikzpicture}
        \end{minipage}%
        \begin{minipage}{0.4\textwidth}
            \centering
            \begin{tabular}{c|ccc}
                Nodes & $a$ & $b$ & $c$ \\
                \hline
                Preds. & $s$ & UD & $a$             
            \end{tabular}
        \end{minipage}
    \end{minipage}

    \underline{Iteration 4:} $\overline{w}_{ba} = 2 + (-4) - (-1) < 0$, so we correct $ba$ by setting $y_a = -2$. The predecessor for $a$ is set to be $b$
    
    \begin{minipage}{\textwidth}
        \centering
        \begin{minipage}{0.6\textwidth}
            \centering
            \begin{tikzpicture}[scale = 0.6]
                \node[vertex][label = left:{$0$}] (s) at (0,0) {$s$};
                \node[vertex][label = above:{$-1$}] (a) at (4,2) {$a$};
                \node[vertex][label = below:{$2$}] (b) at (4,-2) {$b$};
                \node[vertex][label = right:{$4$}] (c) at (8,0) {$c$};
        
                \draw[edge] (s) edge (a);
                \draw[edge] (s) edge (b);
                \draw[edge] (a) edge (c);
                \draw[edge, red] (b) edge (a);
                \draw[edge] (b) edge (c);
            \end{tikzpicture}
        \end{minipage}%
        \begin{minipage}{0.4\textwidth}
            \centering
            \begin{tabular}{c|ccc}
                Nodes & $a$ & $b$ & $c$ \\
                \hline
                Preds. & $s$ & $s$ & $a$             
            \end{tabular}
        \end{minipage}
    \end{minipage}

    \underline{Iteration 5:} $\overline{w}_{ac} = -2 + 5 - 4 < 0$, so we correct $ac$ by setting $y_c = 3$. The predecessor for $c$ stays the same.
    
    \begin{minipage}{\textwidth}
        \centering
        \begin{minipage}{0.6\textwidth}
            \centering
            \begin{tikzpicture}[scale = 0.6]
                \node[vertex][label = left:{$0$}] (s) at (0,0) {$s$};
                \node[vertex][label = above:{$-2$}] (a) at (4,2) {$a$};
                \node[vertex][label = below:{$2$}] (b) at (4,-2) {$b$};
                \node[vertex][label = right:{$4$}] (c) at (8,0) {$c$};
        
                \draw[edge] (s) edge (a);
                \draw[edge] (s) edge (b);
                \draw[edge, red] (a) edge (c);
                \draw[edge] (b) edge (a);
                \draw[edge] (b) edge (c);
            \end{tikzpicture}
        \end{minipage}%
        \begin{minipage}{0.4\textwidth}
            \centering
            \begin{tabular}{c|ccc}
                Nodes & $a$ & $b$ & $c$ \\
                \hline
                Preds. & $b$ & $s$ & $a$             
            \end{tabular}
        \end{minipage}
    \end{minipage}

    And ... we're done! We can check that the red arcs are all equality arcs and that the node potentials are feasible.
    
    \begin{minipage}{\textwidth}
        \centering
        \begin{minipage}{0.6\textwidth}
            \centering
            \begin{tikzpicture}[scale = 0.6]
                \node[vertex][label = left:{$0$}] (s) at (0,0) {$s$};
                \node[vertex][label = above:{$-2$}] (a) at (4,2) {$a$};
                \node[vertex][label = below:{$2$}] (b) at (4,-2) {$b$};
                \node[vertex][label = right:{$3$}] (c) at (8,0) {$c$};
        
                \draw[edge] (s) edge (a);
                \draw[edge, red] (s) edge (b);
                \draw[edge, red] (a) edge (c);
                \draw[edge, red] (b) edge (a);
                \draw[edge] (b) edge (c);
            \end{tikzpicture}
        \end{minipage}%
        \begin{minipage}{0.4\textwidth}
            \centering
            \begin{tabular}{c|ccc}
                Nodes & $a$ & $b$ & $c$ \\
                \hline
                Preds. & $b$ & $s$ & $a$             
            \end{tabular}
        \end{minipage}
    \end{minipage}
\end{example}

\subsubsection{Correctness}
We want to show that the algorithm actually produces a spanning tree rooted at $s$.

Let's first make the following definition.
\begin{definition}{Predecessor Digraph}{}
    Let $D = (N,A)$ be the input digraph for our algorithm. At any point in the algorithm, the \underline{predecessor digraph}, denoted $D_p$, is a digraph with node set $N(D_p) = N$ and arc set $A(D_p) = \{p_vv\::\: v\in N\setminus\{s\}\}$.
\end{definition}

\begin{remark}
    $y_v$ never increases in the algorithm
\end{remark}
And this is since we only modify $y_v$ if we find an arc $uv$ with $y_v > y_u + w_{uv}$. This operation only ever decreases $y_v$

We'll need the following fact in later proofs:
\begin{proposition}{}{}
    Throughout the algorithm, $\overline{w}_e \leq 0$ for all arcs in $D_p$.
\end{proposition}
\begin{proof}
    Focus on an arbitrary node $v$ and its unique predecessor $p_v = u$ (if it exists). When $v$ was corrected, $\overline{w}_{uv} = 0$. Then, until $u$ changes, the reduced cost on $\overline{w}_{uv}$ stays non-positive, \textit{and} only a change in $y_u$ can change the reduced cost on $\overline{w}_{uv}$. By the remark, $y_u$ decreases, so the reduced cost $\overline{w}_{uv} = w_{uv} + y_u - y_v$ also decreases.
\end{proof}

\begin{proposition}{}{}
    If $D_p$ contains a dicycle (at any point in the algorithm), then $D$ contains a negative dicycle and the algorithm does not terminate.
\end{proposition}
\begin{proof}
    Suppose we produce a dicycle $C$ in $D_p$ by correcting the arc $uv$, and let $C$ be $v = v_1,v_2,\ldots,v_k = u,v_1$.

    Since we corrected $uv$, it must have been the case that $y_u + w_{vu} - y_v < 0$. So, $y_{v_k} + w_{v_1v_k} - y_{v_1} < 0$.

    And, we also just showed that $\overline{w}_e \leq 0$ for all $e \in A(D_p)$, so for each $i = 1, 2, \ldots, k -1: \: y_{v_i} + w_{v_iv_{i+1}} - y_{v_{i+1}} < 0$. Taking the sum of these $k$ inequalities, we get:
    $w_{v_1v_2} + \ldots + w_{v_kv_1} < 0$, so $C$ is a negative dicycle.

    And, if there is a negative dicycle, then there is no feasible potential (Try summing up the dual constraints that correspond to the dicycle). As such, the algorithm will never terminate.
\end{proof}

And, there is one more case where the algorithm will not terminate:
\begin{proposition}{}{}
    If $s$ has a predecessor, then $D$ contains a negative dicycle and the algorithm does not terminate.
\end{proposition}
\begin{proof}
    TODO!/Exercise?
\end{proof}

But, in the absence of negative dicycles, the algorithm will terminate. Does it properly give us a spanning tree rooted at $s$?
\begin{proposition}{}{}
    If the algorithm terminates, then $D_p$ is a spanning tree of shortest dipaths rooted at $s$, with $y_v$ as the cost of the shortest $s,v$-dipath.
\end{proposition}
\begin{proof}
    Since the algorithm terminates, $D_p$ does not contain a cycle and $s$ does not have a predecessor. So, $D_p$ is indeed a spanning tree. Further, since all nodes other than $s$ has a predecessor, the in-degree is $1$, so $D_p$ is rooted at $s$.

    Now, all arcs in $D_p$ are equality arcs, since we know $\overline{w}_e \leq 0$ and $\overline{w}_e < 0$ is impossible, since the algorithm has terminated and so $y$ is feasible. 

    For $v \in N$, let $P$ be the $s,v$-dipath in $D_p$. Consider the LP formulation of the shortest $s,v$-dipath problem. $x^P$ is feasible for the primal, and $y$ is feasible for the dual. The CS conditions hold since all arcs in $P$ are equality arcs, so $x^P$ is optimal. The objective of the dual is $y_v - y_s = y_v - 0 = y_v$, so $y_v$ is the cost of the shortest $s,v$-dipath.
\end{proof}