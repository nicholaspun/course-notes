\subsection{Introduction}

We'll now switch gears to the problem of finding shortest dipaths between two nodes $s$ and $t$ in our digraph.
Solving this problem will require some of the tools we've built so far, and we will work our way towards $3$ different algorithms that solve this problem.

\underline{\textbf{Problem: }}Given a digraph $D = (N,A)$, arc costs $w \in R^A$ and $2$ distinct nodes $s,t \in N$, find the minimum cost $s,t$-dipath.

\begin{note}
    Negative arc costs are allowed!
\end{note}

\begin{remark}
    Assume that every node can be reached from $s$ via a dipath.
\end{remark}

\begin{example}{}{intro_st-dipath}
    Below is a digraph with its arcs labelled with its arc cost.

    \begin{minipage}{\textwidth}
        \centering
        \begin{tikzpicture}[scale = 0.75]
            \node[vertex] (s) at (0,0) {s};
            \node[vertex] (a) at (4,2) {a};
            \node[vertex] (b) at (4,-2) {b};
            \node[vertex] (t) at (8,0) {t};
    
            \begin{scope}[every node/.style={fill = white, ellipse}]
                \draw[edge] (s) to node {$3$} (a);
                \draw[edge] (s) to node {$2$} (b);
                \draw[edge] (a) to node {$-2$} (b);
                \draw[edge] (a) to node {$3$} (t);
                \draw[edge] (b) to node {$4$} (t);
            \end{scope}
        \end{tikzpicture}
    \end{minipage}

    $s,a,t$ is a valid dipath of cost $6$.

    $s,b,t$ is also a valid dipath of cost $6$.

    $s,a,b,t$ is a minimum cost dipath of cost $5$.
\end{example}

\underline{\textbf{LP Formulation:}}
\begin{definition}{Characteristic Vector of a Path}{}
    Let $P$ be an $s,t$-dipath. Then, the \underline{characteristic vector} of $P$ is $x^P \in \R^A$ such that
    \[
        x^P_e = \begin{cases}
            1 \:\text{if}\:e\in A(P) \\ 
            0 \:\text{if}\:e\not\in A(P) \\ 
        \end{cases}    
    \]
    (where $A(P)$ is the arc set of $P$)
\end{definition}
We can treat $x^P$ as a flow. We set node $s$ to have demand $-1$ (or supply of $1$), node $t$ to have demand $1$, and all other nodes to have demand $0$.
This suggests the following LP formulation for our problem.
\begin{equation}\label{eq:shortest-dipath_lp-formulation}
\begin{aligned}
    \min \quad &w^\intercal x \\
    \text{s.t} \quad &x(\delta(\overline{v})) - x(\delta(v)) = \begin{cases}
        -1 \:\text{if}\:v=s \\
        1 \:\text{if}\:v=t \\
        0 \:\text{otherwise} \\
    \end{cases}\quad \forall v \in N \\
    &x \geq 0
\end{aligned}
\end{equation}
And, this LP gives the dual:
\begin{equation}\label{eq:shortest-dipath_dual-formulation}
\begin{aligned}
    \max \quad &y_t - y_s \\
    \text{s.t} \quad &y_v - y_u \leq w_{uv} \quad \forall uv \in A \\
    &y\:\text{free}
\end{aligned}
\end{equation}
\begin{note} 
    This LP is just a special case of a TP
\end{note}
We also note that the characteristic vector of any $s,t$-dipath is an integral feasible solution. (Any $s,t$-dipath will send exactly $1$ unit of flow from $s$ through the path until it reaches $t$)

However, the converse of this may not be true!
That is, integral feasible solutions to the LP might not be an $s,t$-dipath
For example, we could consider the following digraph and flow:

\begin{minipage}{\textwidth}
    \centering
    \begin{tikzpicture}
        \node[vertex] (s) at (0,0) {s};
        \node[vertex] (a) at (1.5,2) {a};
        \node[vertex] (b) at (3,0) {b};
        \node[vertex] (c) at (6,0) {c};
        \node[vertex] (t) at (9,0) {t};

        \begin{scope}[every node/.style={fill = white, ellipse}]
            \draw[edge] (s) to node {$6$} (b);
            \draw[edge] (a) to node {$5$} (s);
            \draw[edge] (b) to node {$5$} (a);
            \draw[edge] (b) to node {$3$} (c);
            \draw[edge] (c) to node {$3$} (t);
            \draw[edge, bend right] (t) to node {$2$} (b);
        \end{scope}
    \end{tikzpicture}
\end{minipage}

This flow is a feasible solution to our LP, but we clearly don't have an $s,t$-dipath.

However, this solution \textit{contains} an $s,t$-dipath: $s \rightarrow b \rightarrow c \rightarrow t$. Let's call this $P$. If we subtract $x^P$ from this flow, we obtain the following:

\begin{minipage}{\textwidth}
    \centering
    \begin{tikzpicture}
        \node[vertex] (s) at (0,0) {s};
        \node[vertex] (a) at (1.5,2) {a};
        \node[vertex] (b) at (3,0) {b};
        \node[vertex] (c) at (6,0) {c};
        \node[vertex] (t) at (9,0) {t};

        \begin{scope}[every node/.style={fill = white, ellipse}]
            \draw[edge] (s) to node {$5$} (b);
            \draw[edge] (a) to node {$5$} (s);
            \draw[edge] (b) to node {$5$} (a);
            \draw[edge] (b) to node {$2$} (c);
            \draw[edge] (c) to node {$2$} (t);
            \draw[edge, bend right] (t) to node {$2$} (b);
        \end{scope}
    \end{tikzpicture}
\end{minipage}

This is a set of dicycles!

\begin{theorem}{}{}
    If $\overline{x}$ is an integral feasible solution to our LP ($\Cref{eq:shortest-dipath_lp-formulation}$), then $\overline{x}$ is a sum of the characteristic vector of an $s,t$-dipath and a collection of dicycles.
\end{theorem}
\begin{proof}
    Let $F = \{e \in A\::\overline{x}_e > 0\}$ (since our flow is feasible, this set will be non-empty).
    Let $\delta(S)$ be an $s,t$-cut.
    The net flow of $S$ is $-1$ (since the supply is $-1$), so there must be at least one arc in $\delta(S)$ with non-zero flow, i.e. $|F \cap \delta(S)| \geq 1$.

    So, there is an $s,t$-dipath $P$ using the arcs of $F$. 
    Let $x' = \overline{x} - x^P$. 
    Since $\overline{x}, x^P$ satisfy the flow constraints, then:
    \begin{itemize}
        \item $x'(\delta(\overline{s})) - x'(\delta(s)) = (\overline{x} - x^P)(\delta(\overline{s})) - (\overline{x} - x^P)(\delta(s)) = -1 - (-1) = 0$, and
        \item $x'(\delta(\overline{t})) - x'(\delta(t)) = (\overline{x} - x^P)(\delta(\overline{t})) - (\overline{x} - x^P)(\delta(t)) = 1 - 1 = 0$, and
        \item $x'(\delta(\overline{v})) - x'(\delta(v)) = (\overline{x} - x^P)(\delta(\overline{v})) - (\overline{x} - x^P)(\delta(v)) = 0$, for all other $v \in N$ 
    \end{itemize} 
    So, $x'(\delta(\overline{v})) - x'(\delta(v)) = 0$ for all $v \in N$.
    
    So, we've found our $s,t$-dipath, it remains to show what remains in the flow are all dicycles.

    Let $F' = \{e \in A\::x'_e > 0\}$.
    If $F' = \emptyset$, then we're done.
    So, suppose $F' \neq \emptyset$, and take the longest dipath $v_1\ldots v_k$ in $F'$.

    Since $x'(\delta(\overline{v})) - x'(\delta(v)) = 0$, there is an arc $v_kv_i$ for some $i < k$ (and such $v_i$ cannot be from outside the path since we have taken the longest dipath). This forms a dicycle $C: v_iv_{i+1}\ldots v_kv_i$.

    Then, $x'' = x' - x^C$ satisfies $x''(\delta(\overline{v})) - x''(\delta(v)) = 0$ for all $v \in N$, and the sum of the flow has decreased by at least $1$, so by induction, we are done.
\end{proof} 

So, suppose $\overline{x}$ is an optimal integral solution to our LP.
By this theorem, we can decompose $\overline{x}$ into an $s,t$-dipath and a collection of dicycles, that is: $\overline{x} = x^P + x^{C_1} + \ldots + x^{C_k}$, where $P$ is an $s,t$-dipath and $C_1, \ldots, C_k$ are dicycles.

If $\overline{x} = x^P$, then we're done, so suppose there is at least one dicycle.

If there are \underline{no} negative dicycles, then $w^\intercal \overline{x} \geq w^\intercal x^P$, and so $P$ is an optimal solution in the form of an $s,t$-dipath.
We can find $P$ by using the method described in the proof above.

However, if there are negative dicycles, then our problem is unbounded! (Since we can just send arbitrary amounts of flow through the cycle to decrease our cost).

So, our previous techniques can only help us solve this problem if there are \underline{no} negative dicycles. Let's summarize this.

\begin{theorem}{}{}
    If there are no negative dicycles, then our LP formulation has an optimal solution that is the characteristic vector of an $s,t$-dipath
\end{theorem}

One of our goals in the rest of this section will be to tackle the issue of negative dicycles. Our algorithm should be able to detect that there is a negative dicycle and report it to us.