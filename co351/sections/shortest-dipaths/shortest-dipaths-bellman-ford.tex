\subsection{Bellman-Ford}
Bellman-Ford is a modification of Ford's algorithm that can handle the case of negative dicycles.
As well, we can provide an upper bound on the runtime of this algorithm

\subsubsection{The Algorithm}
The idea for Bellman-Ford is to go through the arcs in (a finite amount of) ``passes''.
In each pass, we perform the same action as with Ford's algorithm.

\IncMargin{1em}
\begin{algorithm}[H]\label{alg:bellman-ford}  
  \nl \textbf{Initialization:}
  \begin{itemize}
      \item Set $y_s = 0$ and $y_v = \infty\:\forall v \in N\setminus\{s\}$ 
      \item Set $p_v =$ undefined $\forall v \in N$
      \item Set $i \leftarrow 0$
  \end{itemize}

  \nl \While{$i < |N| - 1$}{
    \begin{enumerate}[label=(\alph*)]
        \item For each arc $uv \in A$, if $\overline{w}_{uv} < 0$, then set $y_v = y_u + w_{uv}$ and $p_v = u$
        \item Set $i \leftarrow i + 1$
    \end{enumerate}
  }

  \caption{Bellman-Ford Algorithm}
\end{algorithm}

The while loops runs a maximum of $|N|$ times, and in each iteration, we check every arc in $A$. So in total the runtime of this algorithm is: $\mathcal{O}(|N||A|)$.

If a feasible potential is found once the algorithm terminates, then everything we've proved about Ford's algorithm applies and so we have our optimal rooted tree.

However, if we still have an infeasible potential by termination, then there is a negative dicycle in our digraph.

\subsubsection{Correctness}
\underline{Notation:} We'll use $d_v$ to denote the cost of a shortest $s,v$-dipath.

First, we'll show a lemma:
\begin{lemma}{}{}
    Suppose there are no negative dicycles. Then, at any point in the algorithm $y_v \geq d_v$
\end{lemma}
\begin{proof}
    This clearly holds at initialization since $\infty \geq d_v$ (and $d_v$ is finite).

    If $y_v \neq\infty$, then we can trace $v$ back to $s$ using $D_p$. For each of these arcs $\overline{w}_e \leq 0$. Adding all of these inequalities for each arc in this $s,v$-dipath, we get: 
    \begin{equation*}
        \sum_{ab \in P} \overline{w}_{ab} = \sum_{ab \in P} w_{ab} + y_a - y_b = w(P) - y_v \leq 0 \Rightarrow w(P) \leq y_v
    \end{equation*}
    But, $w(P) \geq d_v$, since $d_v$ is the cost of the a shortest $s,v$-dipath, so $d_v \leq y_v$, as required.
\end{proof}

\begin{theorem}{}{}
    Suppose there are no negative dicycles. After the $i$-th iteration, if there is a shortest $s,v$-dipath using at most $i$ arcs, then $y_v = d_v$
\end{theorem}
\begin{proof}
    We'll prove this by induction on $i$. The statement holds for $i = 0$. Assume the statement holds after the $i$-th iteration.

    We want to show that the statement holds for $(i+1)$-th iteration. Pick $v$ which has a shortest $s,v$-dipath that uses at most $i+1$ arcs.

    If the dipath uses at most $i$ arcs, then by induction, $y_v = d_v$ and by lemma, $y_v$ will not change.

    So, suppose the shortest $s,v$-dipath uses exactly $i+1$ arcs and let $P: s=v_1\ldots v_{i+1} =v$. Since there are no negative dicycles, $v_1\ldots v_i$ is a shortest $s,v_i$-dipath that uses $i$ arcs and by induction, $y_{v_i} = d_{v_i}$ after the $i$-th iteration. This does not change in the $(i+1)$-th iteration.

    Consider  $\overline{w}_{v_iv_{i+1}}$, there are three cases:
    \begin{enumerate}
        \item $\overline{w}_{v_iv_{i+1}} = 0$: Then, $y_{v_{i+1}} = y_{v_i} + w_{v_iv_{i+1}} = d_{v_i} + w_{v_iv_{i+1}} = w(P) = d_{v_{i+1}}$. The arc was already corrected before the $(i+1)$-th iteration, and will not change in this iteration (or future ones).
        \item $\overline{w}_{v_iv_{i+1}} > 0$: Then, $y_{v_{i+1}} < d_{v_{i+1}}$. But, this contradicts the lemma, so this case does not happen
        \item $\overline{w}_{v_iv_{i+1}} < 0$: Then, in the $(i+1)$-th iteration, the algorithm corrects this arc, and so $y_{v_{i+1}} = d_{v_{i+1}}$.
    \end{enumerate}
    By induction, we're done!
\end{proof}

\begin{corollary}{}{}
    At the end of Bellman-Ford, if $y$ is feasible, then $y_v = d_v$ for all $v \in N$. Otherwise, you can conclude there is a negative dicycle.
\end{corollary}
\begin{proof}
    Any shortest dipath uses at most $|N| - 1$ arcs. So, if $y$ is feasible by the end of the $|N| -1$ passes in Bellman-Ford, there must be no negative dicycles and by theorem $y_v = d_v \:\forall v \in N$. If $y$ is not feasible, then it must be the case that there is a negative dicycle.
\end{proof}