\subsection{Introduction}
For the maximum flow problem, we are given a digraph $D = (N,A)$, arc capacities $c \in \R^A$, and a source and sink node $s,t \in N$.
The goal is to push as much flow as we can from $s$ to $t$.

\underline{LP Formulation:}

As always, we'll formulate the LP for this problem:
\begin{equation}\label{eq:max-flow_lp-formulation}
\begin{aligned}
      \max \quad &x(\delta(s)) - x(\delta(\overline{s}))\\
      \text{s.t} \quad &x(\delta(\overline{v})) - x(\delta(v)) = 0 \quad&\forall v \in N \setminus \{s, t\}\\
      &x_e \leq c_e \quad &\forall e \in A \\
      &x \geq 0
\end{aligned}
\end{equation}
Our objective function reflects our goal by maximizing flow out of $s$.

\underline{Exercise:} Formulate this as an MCFP. (TODO)

There are 2 main classes of algorithms for this problem: the augmenting-path based algorithms and the ``push-relabel'' type algorithms.
We will go over both types. 
Further, in this section, we will derive and prove the \textit{Max Flow, Min Cut Theorem} and apply it towards some interesting combinatorial results.
