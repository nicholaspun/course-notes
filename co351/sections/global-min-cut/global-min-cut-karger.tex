\subsection{Karger's Randomized Algorithm}
We'll now consider minimum cuts in undirected graphs.
The setup here is: Given an undirected graph $G = (V, E)$ and edge capacities $c \in \R_+^E$, we want to find the minimum capacity non-trivial cut in $G$.

\begin{example}{}{}
    Suppose we are given the following graph:

    \begin{minipage}{\textwidth}
        \centering
        \begin{tikzpicture}
            \node[vertex] (a) at (0,0) {$a$};
            \node[vertex] (b) at (0,3) {$b$};
            \node[vertex] (d) at (3,3) {$d$};
            \node[vertex] (c) at (3,0) {$c$};

            \begin{scope}[every node/.style={fill = white, ellipse}]
                \draw (a) to node {$2$} (b);
                \draw (b) to node {$3$} (c);
                \draw (b) to node {$7$} (d);
                \draw (d) to node {$4$} (c);
                \draw (a) to node {$8$} (c);
            \end{scope}
        \end{tikzpicture}
    \end{minipage}

    The cut $\delta(\{a\})$ has capacity $c(\delta(\{a\})) = 2 + 8 = 10$.

    The cut $\delta(\{a, c\})$ has capacity $c(\delta(\{a, c\})) = 2 + 3 + 4 = 10$.
\end{example}

We first define the operation of edge contraction:
\begin{definition}{Edge Contraction}{}
    A contraction on edge $e = uv$ is removing the edge $e$, then merging $u$ and $v$ into one vertex $w$ and the edges incident to $u$ or $v$ are now incident to $w$
\end{definition}
\begin{example}{Edge Contraction}{}
    An edge contraction of the edge $ab$ on the left graph gives the graph on the right:

    \begin{minipage}{0.5\textwidth}
        \centering
        \begin{tikzpicture}[scale=0.8]
            \node[vertex] (a) at (0,0) {$a$};
            \node[vertex] (b) at (0,3) {$b$};
            \node[vertex] (d) at (3,3) {$d$};
            \node[vertex] (c) at (3,0) {$c$};

            \begin{scope}[every node/.style={fill = white, ellipse}]
                \draw (a) to node {$2$} (b);
                \draw (b) to node {$3$} (c);
                \draw (b) to node {$7$} (d);
                \draw (d) to node {$4$} (c);
                \draw (a) to node {$8$} (c);
            \end{scope}
        \end{tikzpicture}
        \captionof*{figure}{Original Graph}
    \end{minipage}
    \begin{minipage}{0.5\textwidth}
        \centering
        \begin{tikzpicture}[scale=0.8]
            \node[vertex] (ab) at (0,1.5) {$ab$};
            \node[vertex] (d) at (3,3) {$d$};
            \node[vertex] (c) at (3,0) {$c$};

            \begin{scope}[every node/.style={fill = white, ellipse}]
                \draw[bend left] (ab) to node {$3$} (c);
                \draw (ab) to node {$7$} (d);
                \draw (d) to node {$4$} (c);
                \draw[bend right] (ab) to node {$8$} (c);
            \end{scope}
        \end{tikzpicture}
        \captionof*{figure}{Resulting Graph}
    \end{minipage}
\end{example}

The \underline{idea} for Karger's algorithm is to repeatedly contract edges until only $2$ vertices remain.
The edges between the remaining two super-vertices represents a cut (and for some probability, it may also be a minimum cut).

\IncMargin{1em}
\begin{algorithm}[H]\label{alg:karger}  
    \nl \While{$|V| > 2$}{
        \begin{enumerate}[label=(\alph*)]
            \item Pick an edge $uv$ with probability $\frac{c_{uv}}{\sum_{e \in E} c_e}$
            \item Contract $uv$
        \end{enumerate}
    }

    \nl Let $(S, V\setminus S)$ be the vertex sets corresponding to the $2$ vertices remaining. \Return $\delta(S)$ 

    \caption{Karger's Algorithm}
\end{algorithm}

This is a randomized algorithm, so we want to ask how often the resulting cut will be a minimum cut. We'll need the following remark:
\begin{remark}{}{}
    Let $G = (V, E)$ be an undirected graph.
    \begin{equation*}
        \sum_{e \in E} c_e = \frac{1}{2}\sum_{v \in V} c(\delta(\{v\}))
    \end{equation*} 
\end{remark}
\begin{proof}
    Every edge $uv$ appears in $2$ such cuts: $\delta(\{u\})$ and $\delta(\{v\})$. The result follows.
\end{proof}

And now to the main result:
\begin{theorem}{}{}
    Let $\delta(S)$ be a specific minimum cut. The probability that the algorithm produces $\delta(S)$ is at least $\frac{1}{\binom{|V|}{2}}$
\end{theorem}
\begin{proof}
    Consider the $(k + 1)$-th iteration of our algorithm when we have $|V| - k$ vertices remaining. Assume that we have not yet contracted any edge in $\delta(S)$ in iteration $1, \ldots, k$. Call our current graph $G' = (V', E')$. The probability of selecting an edge in $\delta(S)$ is:
    \begin{align*}
        \frac{c(\delta(S))}{\sum_{e \in E'} c_e} &= \frac{c(\delta(S))}{\sum_{v \in V'} c(\delta(\{v\}))} \quad &(\text{By remark})\\
        &\leq \frac{c(\delta(S))}{\sum_{v \in V'} c(\delta(S))} \quad &(\text{Since } \delta(S) \text{ is a minimum cut}) \\
        &= \frac{c(\delta(S))}{\frac{|V| - k}{2}c(\delta(S))} \\
        &= \frac{2}{|V| - k}
    \end{align*}

    So the probabilty that $\delta(S)$ survives this iteration is at least $1 - \frac{2}{|V| - k}$

    The largest possible $k$ is $|V| - 3$ (since the algorithm terminates when there are only $2$ vertices remaining).
    Overall, the probability that $\delta(S)$ survives every contraction is at least:
    \begin{align*}
        &\prod_{k = 0}^{|V| - 3} (1 - \frac{2}{|V| - k}) \\
        &= \prod_{k = 0}^{|V| - 3} (1 - \frac{|V| - k - 2}{|V| - k}) \\
        &= \frac{|V| - 2}{|V|}\frac{|V| - 3}{|V| - 1}\frac{|V| - 4}{|V| - 2}\ldots\frac{2}{4}\frac{1}{3} \\
        &= \frac{2}{|V|(|V| - 1)} \\
        &= \frac{1}{\binom{|V|}{2}}
    \end{align*}
\end{proof}

This probability is not too low. But, if we were to only run the algorithm once, we would most likely not obtain any minimum cut!

To boost this probability, we want to re-run the algorithm at least $\alpha|V|^2$, and output the smallest capacity cut that we find.
\begin{corollary}{}{}
    The probability that the algorithm produces a specific minimum cut $\delta(S)$ after $\alpha|V|^2$ runs is at least $1 - e^{-2\alpha}$ $(\alpha \geq 1)$
\end{corollary}
\begin{proof}
    We'll use the inequality $1 - x \leq e^{-x}$.

    The probability of not producing $\delta(S)$ is at most:
    \begin{align*}
        \left(1 - \frac{1}{\binom{|V|}{2}}\right)^{\alpha|V|^2} &\leq \left(1 - \frac{2}{|V|^2}\right)^{\alpha|V|^2} \\
        &\leq \left(e^{-\frac{2}{|V|^2}}\right)^{\alpha|V|^2} \\
        &= e^{-2k}
    \end{align*}

    And so our desired result follows.
\end{proof}