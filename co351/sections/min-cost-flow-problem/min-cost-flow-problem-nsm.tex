\subsection{Network Simplex Method for MCFP}
Much like with the transshipment problem, we'll develop a similar Network Simplex Method for MCFP.

\subsubsection{Finding a Basis}
Again, we start by asking the question: \textit{What does a basis look like for MCFP?}

Let's put our constraints from our LP in \Cref{eq:mcfp_lp-formulation} into matrix form. 
We'll introduce $|A|$ slack variables $s_e$ to get rid of the inequality in the 2nd constraint ($x_e \leq c_e$)
\[
    \renewcommand{\arraystretch}{2}
    \begin{blockarray}{ccccc}
        & x_e & s_e & &\\
    \begin{block}{c[c|c]c[c]}
        V & M_{|V| \times |A|} & 0_{|V| \times |A|} & \multirow{2}{*}{=} & b_v\\
        \cline{2-3}
        A & I_{|A| \times |A|} & I_{|A| \times |A|} & & c_e \\
    \end{block}
    \end{blockarray}
\]
($M$ is the node-arc incidence matrix, $I$ is the identity matrix, and $0$ is the matrix of all zeroes)

We know that $M$ has a rank of $|N| - 1$. 
We'll extend this by putting all of the other constraints into our basis.
There are $|A|$ remaining constraints, so the total rank of our matrix is $|N| - 1 + |A|$.
That means that a basis for the MCFP will have $|N| - 1 + |A|$ variables, and all non-basic variables will be set to $0$.

Observe that for each arc $e \in A$, at least one of $x_e$ or $s_e$ is in the basis. Why?
If both $x_e = s_e = 0$, then $x_e + s_e = 0 < c_e$, since we assume that $c_e$ are all positive.
So, this violates feasibility.

Therefore, for each arc $e \in A$, we have the following $3$ possibilities:
\begin{enumerate}
    \item Only $x_e$ is in the basis (and $s_e$ is not)
    \item Only $s_e$ is in the basis
    \item Both $x_e$ and $s_e$ are in the basis.
\end{enumerate}

How many arcs can there be in case $3$?
Let's try to count them; denote the number of arcs in case $3$ with $k$.
Then, there are $|A| - k$ arcs in case $1$ or $2$.
This gives a total of $2k + |A| - k = k + |A|$ variables.
Our basis has size $|N| + |A| - 1$, so $k = |N| - 1$.
\begin{remark}
    These $|N| - 1$ arcs don't contain a cycle.
\end{remark}
\begin{proof}
Exercise! (This number should seem awfully similar to something we've shown before)
\end{proof}
So, these $|N| - 1$ arcs from case $3$ must represent a spanning tree!

\underline{\textbf{In Summary:}}

A basic feasible solution in MCFP consists of a spanning tree $T$  and a feasible flow where all arcs not in $T$ satisfies $x_e \in \{0, c_e\}$. (Since if the arc falls under case $1$, then only $x_e$ is in basis and so $s_e = 0 \Rightarrow x_e = c_e$, and similarly for arcs under case $2$)

\begin{example}{}{}
Below is a feasible flow (labelled with $x_e \leq c_e$). The dashed lines are in the spanning tree. So, the variables in the basis are: $x_{ab}, x_{bd}, x_{ac}, s_{ac}, x_{bc}, s_{bc}, x_{cd}, s_{cd}$.

\begin{minipage}{\textwidth}
    \centering
    \begin{tikzpicture}[scale = 0.75]
        \node[vertex][label=left:{$-10$}] (a) at (0,0) {a};
        \node[vertex][label={$-5$}] (b) at (4.5,3) {b};
        \node[vertex][label=below:{$12$}] (c) at (4.5,-3) {c};
        \node[vertex][label=right:{$3$}] (d) at (9,0) {d};
        
        \begin{scope}[every node/.style={fill = white, ellipse}]
            \draw[edge] (a) to node {$7 \leq 7$} (b);
            \draw[edge, dashed] (a) to node {$3 \leq 5$} (c);
            \draw[edge] (b) to node {$2 \leq 2$} (d);
            \draw[edge, dashed] (b) to node {$10 \leq 12$} (c);
            \draw[edge, dashed] (c) to node {$1 \leq 8$} (d);
        \end{scope}
    \end{tikzpicture}
\end{minipage}
\end{example}

\subsubsection{Optimality Conditions}
Good, now we know what a basic feasible solution looks like for MCFP, we want to be able to improve it until it's optimal.
Again, we'll use the CS conditions to derive optimal conditions.

Using \Cref{eq:mcfp_lp-formulation,eq:mcfp_dual-formulation}, we obtain the following 2 CS conditions:
\begin{enumerate}
    \item $z_{uv} > 0 \Rightarrow x_{uv} = c_{uv}$
    \item $-y_u + y_v - z_{uv} < w_{uv} \Rightarrow x_{uv} = 0$
\end{enumerate}

There's a lot of variables here!
Let's try to rewrite these conditions in terms reduced costs. (Recall that the reduced cost for an arc $uv$ is $\overline{w}_{uv} = y_u + w_{uv} - y_v$)

Then, the dual constraint: $-y_u + y_v - z_{uv} \leq w_{uv}$ can be rewritten as $z_{uv} \geq -\overline{w}_{uv}$, and we'll keep the second constraint: $z_{uv} \geq 0$ as is.
Note that in the dual LP, our objective function maximizes $b^\intercal y - c^\intercal z$.
In particular, we want to minimize $c^\intercal z$.
So, $z_{uv} = \max\{0, -\overline{w}_{uv}\}$

Using this, we can rewrite our two CS conditions:
\begin{enumerate}
    \item We had $z_{uv} > 0 \Rightarrow x_{uv} = c_{uv}$, but $z_{uv} > 0$ if and only if $\overline{w}_{uv} < 0$. So, this is equivalent to saying:
    \begin{equation*}
        \overline{w}_{uv} < 0 \Rightarrow x_{uv} = c_{uv}
    \end{equation*}

    \item We had $-y_u + y_v - z_{uv} < w_{uv} \Rightarrow x_{uv} = 0$. But, we can rewrite the premise as $z_{uv} > -\overline{w}_{uv}$ and this is true if and only if $\overline{w}_{uv} > 0$. So, this condition is equivalent to:
    \begin{equation*}
        \overline{w}_{uv} > 0 \Rightarrow x_{uv} = 0
    \end{equation*} 
\end{enumerate}

\underline{\textbf{In Summary:}}
\begin{enumerate}
    \item $\overline{w}_{uv} < 0 \Rightarrow x_{uv} = c_{uv}$
    \item $\overline{w}_{uv} > 0 \Rightarrow x_{uv} = 0$
\end{enumerate}
And, if $0 < x_{uv} \leq c_{uv}$, then $\overline{w}_uv = 0$.

To gain some intuition for the optimality conditions, recall that we introduced an economic interpretation for NSM for TPs in \Cref{sec:tp_econ-interp}. 
We can extend the interpretation by capping the amount we can transport through an arc $uv$ by $c_{uv}$.
Then, using the same interpretation of the reduced cost, when $\overline{w}_{uv} < 0$, we can make profit by using arc $uv$, so we should send as much as possible through $uv$.
The maximum we can send is $c_{uv}$, so set $x_{uv} = c_{uv}$.

And, if $\overline{w}_{uv} > 0$, we lose money through $uv$, so we don't send anything through this arc, i.e. $x_{uv} = 0$

\begin{example}{Checking for optimality}{}
    On the left is a digraph with arc labelled by $(x_e, c_e)$ and on the right is a feasible flow along with feasible node potentials. The dashed lines are in the spanning tree.

    We can check that $\overline{w}_{bc} = w_{bc} + y_b - y_c = -110 < 0$, but $x_{bc} = 10 \neq 12$, so this flow is not optimal

    \begin{minipage}{0.5\textwidth}
        \centering
        \begin{tikzpicture}[scale = 0.6]
            \node[vertex][label=left:{$-10$}] (a) at (0,0) {a};
            \node[vertex][label={$-5$}] (b) at (4.5,3) {b};
            \node[vertex][label=below:{$12$}] (c) at (4.5,-3) {c};
            \node[vertex][label=right:{$3$}] (d) at (9,0) {d};
            
            \begin{scope}[every node/.style={fill = white, ellipse}]
                \draw[edge] (a) to node {$(50,7)$} (b);
                \draw[edge] (a) to node {$(70,5)$} (c);
                \draw[edge] (b) to node {$(20,2)$} (d);
                \draw[edge] (b) to node {$(30,12)$} (c);
                \draw[edge] (c) to node {$(100,8)$} (d);
            \end{scope}
        \end{tikzpicture}
    \end{minipage}
    \begin{minipage}{0.5\textwidth}
        \centering
        \begin{tikzpicture}[scale = 0.6]
            \node[vertex][label=left:{$0$}] (a) at (0,0) {a};
            \node[vertex][label={$50$}] (b) at (4.5,3) {b};
            \node[vertex][label=below:{$80$}] (c) at (4.5,-3) {c};
            \node[vertex][label=right:{$180$}] (d) at (9,0) {d};
            
            \begin{scope}[every node/.style={fill = white, ellipse}]
                \draw[edge, dashed] (a) to node {$5$} (b);
                \draw[edge] (a) to node {$5$} (c);
                \draw[edge] (b) to node {$10$} (d);
                \draw[edge, dashed] (b) to node {$10$} (c);
                \draw[edge, dashed] (c) to node {$3$} (d);
            \end{scope}
        \end{tikzpicture}
    \end{minipage}
\end{example}

\subsubsection{Entering and Leaving Arcs}
Now, for the final step in creating NSM for MCFP.

For entering arcs, we should choose arcs that violate either optimality condition (which we'll quickly recall):
\begin{enumerate}
    \item $\overline{w}_{uv} < 0 \Rightarrow x_{uv} = c_{uv}$
    \item $\overline{w}_{uv} > 0 \Rightarrow x_{uv} = 0$
\end{enumerate}

If the reduced cost $\overline{w}_{uv} < 0$, but $x_{uv} < c_{uv}$, then there's nothing to modify here.
We proceed as we did with NSM for TP: Create a cycle using the arc $uv$, and push as much flow through $uv$ as we can. 
The arc that leaves the basis is the one that reaches capacity or is reduced to $0$ flow.

However, if the reduced cost $\overline{w}_{uv} > 0$, we want to make $x_{uv} = 0$.
In this case, we want to \textit{decrease} flow through $uv$.
So, instead of orienting the cycle in the direction of $uv$, let's orient the cycle in the reverse direction of $uv$. The leaving arc is, again, the arc that reaches capacity or is reduced to $0$ flow.

\begin{example}{Removing flow from the entering arc}{}
    On the left is a digraph with arc labelled by $(x_e, c_e)$ and on the right is a feasible flow along with feasible node potentials. The dashed lines are in the spanning tree.

    \begin{minipage}{0.5\textwidth}
        \centering
        \begin{tikzpicture}[scale = 0.6]
            \node[vertex][label=left:{$-10$}] (a) at (0,0) {a};
            \node[vertex][label={$-5$}] (b) at (4.5,3) {b};
            \node[vertex][label=below:{$12$}] (c) at (4.5,-3) {c};
            \node[vertex][label=right:{$3$}] (d) at (9,0) {d};
            
            \begin{scope}[every node/.style={fill = white, ellipse}]
                \draw[edge] (a) to node {$(50,7)$} (b);
                \draw[edge] (a) to node {$(70,5)$} (c);
                \draw[edge] (b) to node {$(20,2)$} (d);
                \draw[edge] (b) to node {$(30,12)$} (c);
                \draw[edge] (c) to node {$(100,8)$} (d);
            \end{scope}
        \end{tikzpicture}
    \end{minipage}
    \begin{minipage}{0.5\textwidth}
        \centering
        \begin{tikzpicture}[scale = 0.6]
            \node[vertex][label=left:{$0$}] (a) at (0,0) {a};
            \node[vertex][label={$40$}] (b) at (4.5,3) {b};
            \node[vertex][label=below:{$70$}] (c) at (4.5,-3) {c};
            \node[vertex][label=right:{$170$}] (d) at (9,0) {d};
            
            \begin{scope}[every node/.style={fill = white, ellipse}]
                \draw[edge] (a) to node {$7$} (b);
                \draw[edge, dashed] (a) to node {$3$} (c);
                \draw[edge] (b) to node {$2$} (d);
                \draw[edge, dashed] (b) to node {$10$} (c);
                \draw[edge, dashed] (c) to node {$1$} (d);
            \end{scope}
        \end{tikzpicture}
    \end{minipage}

    We want to calculate the reduced costs on $bd$ and $ab$.
    \begin{itemize}
        \item $\overline{w}_{bd} =  20 -170 + 40 = -110 < 0$ and $x_{bd} = c_{bd}$, so this arc is fine.
        \item $\overline{w}_{ab} =  50 - 40 = 10 > 0$, but $x_{ab} \neq 0$, so this arc violates the optimality conditions.
    \end{itemize}

    So, $ab$ is our entering arc, and we want to \textit{decrease} the flow through this arc. Orient $ab$ as a reverse arc in the cycle $acb$, and let $t$ be the flow we push through the cycle

    \begin{minipage}{\textwidth}
        \centering
        \begin{tikzpicture}[scale = 0.6]
            \node[vertex][label=left:{$0$}] (a) at (0,0) {a};
            \node[vertex][label={$40$}] (b) at (4.5,3) {b};
            \node[vertex][label=below:{$70$}] (c) at (4.5,-3) {c};
            \node[vertex][label=right:{$170$}] (d) at (9,0) {d};
            
            \begin{scope}[every node/.style={fill = white, ellipse}]
                \draw[edge, red] (a) to node {$7 - t$} (b);
                \draw[edge, dashed, red] (a) to node {$3 + t$} (c);
                \draw[edge] (b) to node {$2$} (d);
                \draw[edge, dashed, red] (b) to node {$10 - t$} (c);
                \draw[edge, dashed] (c) to node {$1$} (d);
            \end{scope}
        \end{tikzpicture}
    \end{minipage}

    We can take away at most $7$ and $10$ flow from $ab$ and $bc$ respectively. We can add at most $2$ flow to $ac$, since its capacity is $5$. So, $t = \min\{7, 10, 2\} = 2$. $ac$ reaches capacity so it leaves the basis and $ab$ enters.

    \begin{minipage}{\textwidth}
        \centering
        \begin{tikzpicture}[scale = 0.6]
            \node[vertex][label=left:{$0$}] (a) at (0,0) {a};
            \node[vertex][label={$40$}] (b) at (4.5,3) {b};
            \node[vertex][label=below:{$70$}] (c) at (4.5,-3) {c};
            \node[vertex][label=right:{$170$}] (d) at (9,0) {d};
            
            \begin{scope}[every node/.style={fill = white, ellipse}]
                \draw[edge, dashed, red] (a) to node {$5$} (b);
                \draw[edge, red] (a) to node {$5$} (c);
                \draw[edge] (b) to node {$2$} (d);
                \draw[edge, dashed, red] (b) to node {$8$} (c);
                \draw[edge, dashed] (c) to node {$1$} (d);
            \end{scope}
        \end{tikzpicture}
    \end{minipage}

    And, one can check that this flow is optimal!
\end{example}

\subsubsection{The Algorithm}

\IncMargin{1em}
\begin{algorithm}[H]\label{alg:nsm-mcfp}
  \SetKwInOut{Input}{Input}

  \Input{  
    \begin{itemize}
      \item $D = (N,A)$ connected digraph
      \item $w \in R^A$ arc costs
      \item $b \in R^N$ node demands
      \item $c \in R^A$ arc capacities
    \end{itemize}
  }

  \BlankLine
  
  \nl Look for an arc such that
  \begin{enumerate}
      \item $\overline{w}_{e} < 0$ and $x_e = 0$, or 
      \item $\overline{w}_{e} > 0$ and $x_e = c_e$
  \end{enumerate}

  \nl $T + e$ has a cycle $C$. Depending on which case $e$ falls under:
  \begin{enumerate}
      \item Orient $C$ such that $e$ is a forward arc
      \item Orient $C$ such that $e$ is a backward arc
  \end{enumerate}

  \nl Find $t$ that is the minimum of: $\{c_f - x_f\::\:f\:\text{is a forward arc}\}$ and $\{x_f\::\:f\:\text{is a backward arc}\}$

  \nl Pick $f$ to be an arc in $C$ that is used to pick $t$

  \nl Update the flow:
  \begin{itemize}
      \item Add $t$ to forward arcs in the cycle,
      \item Subtract $t$ from backward arcs
  \end{itemize}

  \nl Replace $T$ with $T + e - f$

  \nl Repeat until optimal

  \caption{Network Simplex Algorithm For MCFP}
\end{algorithm}
