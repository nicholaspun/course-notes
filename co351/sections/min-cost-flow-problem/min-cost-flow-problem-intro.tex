
\subsection{Introduction}
The \textit{Minimum Cost Flow Problem} (MCFP) is quite similar to the transhippment problem. 
We are given:
\begin{itemize}
    \item A digraph $D = (N,A)$,
    \item Node demands $b \in \R^N$, 
    \item Arc costs $w \in R^A$, and
    \item (the new addition) Arc capacities $c \in R^A$, representing the maximum amount of flow that can be pushed through an arc. (We'll assume that $c_e > 0$, since otherwise, we can just remove the arc)
\end{itemize}

Our \underline{goal} is for this problem is to find a feasible flow (if it exists) satisfying the node demands \textit{and arc capacities}, while minimizing total cost.

\begin{example}{}{}
    Below is an example of an MCFP. Arcs are labelled with a pair $(w_e, c_e)$---the first value is the arc cost, and the second is the arc capacity.

    \begin{minipage}{\textwidth}
        \centering
        \begin{tikzpicture}[scale = 0.75]
            \node[vertex][label=left:{$-10$}] (a) at (0,0) {a};
            \node[vertex][label={$-5$}] (b) at (4.5,3) {b};
            \node[vertex][label=below:{$12$}] (c) at (4.5,-3) {c};
            \node[vertex][label=right:{$3$}] (d) at (9,0) {d};
            
            \begin{scope}[every node/.style={fill = white, ellipse}]
                \draw[edge] (a) to node {$(50,7)$} (b);
                \draw[edge] (a) to node {$(70,5)$} (c);
                \draw[edge] (b) to node {$(20,2)$} (d);
                \draw[edge] (b) to node {$(30,12)$} (c);
                \draw[edge] (c) to node {$(100,8)$} (d);
            \end{scope}
        \end{tikzpicture}
    \end{minipage}
    And, this example has a feasible solution:

    \begin{minipage}{\textwidth}
        \centering
        \begin{tikzpicture}[scale = 0.75]
            \node[vertex][label=left:{$-10$}] (a) at (0,0) {a};
            \node[vertex][label={$-5$}] (b) at (4.5,3) {b};
            \node[vertex][label=below:{$12$}] (c) at (4.5,-3) {c};
            \node[vertex][label=right:{$3$}] (d) at (9,0) {d};
            
            \begin{scope}[every node/.style={fill = white, circle}]
                \draw[edge] (a) to node {$7$} (b);
                \draw[edge] (a) to node {$3$} (c);
                \draw[edge] (b) to node {$2$} (d);
                \draw[edge] (b) to node {$10$} (c);
                \draw[edge] (c) to node {$1$} (d);
            \end{scope}
        \end{tikzpicture}
    \end{minipage}
\end{example}

We can extend our LP formulation from TP for MCFP by including the arc capacity constraints:
\begin{equation}\label{eq:mcfp_lp-formulation}
\begin{aligned}
      \min \quad &w^\intercal x \\
      \text{s.t} \quad &x(\delta(\overline{v})) - x(\delta(v)) = b_v \quad \forall v \in N \\
      &x_e \leq c_e \quad \forall e \in A \\
      &x \geq 0
\end{aligned}
\end{equation}

The dual of this LP is:
\begin{equation}\label{eq:mcfp_dual-formulation}
\begin{aligned}
    \max \quad &b^\intercal y - c^\intercal z \\
    \text{s.t} \quad &-y_u + y_v - z_{uv} \leq w_{uv} \quad \forall uv \in A \\
    &z_{uv} \geq 0 \quad \forall uv \in A \\
    &y\:\text{free}
\end{aligned}
\end{equation}

In this section, we'll tweak the NSM algorithm to solve MCFP instances and find an infeasibility characterization for the problem.
Note that this problem cannot be unbounded since $x_e \leq c_e$ places an upper bound on the maximum flow we can push.
