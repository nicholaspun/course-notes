\subsection{Infeasibility}\label{sec:tp_infeasbility-characterization}

\textbf{Question: }How do we find an initial tree flow?

As promised, we will discuss how to find an initial tree flow in this section. Turns out, this will help us find an infeasibility characterization.

Recall that in the simplex method, we find our initial solution using an auxiliary problem. 
We will create an auxiliary TP here.
We know that every node with negative demands (the supply nodes) supply $b_v$ units of flow.
Let's create a new node $z$ that accepts all the flow from the supply nodes.
Now, since we assume $\sum b_v = 0$, $z$ has just enough excess flow to push towards the demand nodes.
We demonstrate this below:

\begin{minipage}{0.5\textwidth}
    \centering
    \begin{tikzpicture}[scale = 0.75]
        \node[vertex][label=left:{$-5$}] (a) at (0,0) {a};
        \node[vertex][label={$7$}] (b) at (4,2) {b};
        \node[vertex][label=below:{$0$}] (c) at (4,-2) {c};
        \node[vertex][label=right:{$-5$}] (d) at (8,0) {d};

        \draw[edge] (a) to node {} (b);
        \draw[edge] (a) to node {} (c);
        \draw[edge] (b) to node {} (d);
        \draw[edge] (c) to node {} (b);
        \draw[edge] (c) to node {} (d);
    \end{tikzpicture}
    \captionof*{figure}{Original Digraph}
\end{minipage}
\begin{minipage}{0.5\textwidth}
    \centering
    \begin{tikzpicture}[scale = 0.75]
        \node[vertex][label=left:{$-5$}] (a) at (0,0) {a};
        \node[vertex][label={$7$}] (b) at (4,2) {b};
        \node[vertex][label=below:{$0$}] (c) at (4,-2) {c};
        \node[vertex][label=right:{$-5$}] (d) at (8,0) {d};
        \node[vertex][label=above:{$3$}] (z) at (8,4) {z};

        \draw[edge] (a) to node {} (b);
        \draw[edge] (a) to node {} (c);
        \draw[edge] (b) to node {} (d);
        \draw[edge] (c) to node {} (b);
        \draw[edge] (c) to node {} (d);
        \draw[edge, bend left] (a) to node {} (z);
        \draw[edge] (b) to node {} (z);
        \draw[edge] (z) to node {} (d);
        \draw[edge, out = 0, in = 0, looseness = 2] (z) to node {} (c);
    \end{tikzpicture}
    \captionof*{figure}{Digraph with our new node}
\end{minipage}

We want any solution to this modified digraph \underline{not} use any of the new arcs, so let's set the cost of the original arcs to be $0$ and the new arcs to be $1$

\begin{minipage}{\textwidth}
    \centering
    \begin{tikzpicture}[scale = 0.75]
        \node[vertex][label=left:{$-5$}] (a) at (0,0) {a};
        \node[vertex][label={$7$}] (b) at (4,2) {b};
        \node[vertex][label=below:{$0$}] (c) at (4,-2) {c};
        \node[vertex][label=right:{$-5$}] (d) at (8,0) {d};
        \node[vertex][label=above:{$3$}] (z) at (8,4) {z};

        \begin{scope}[every node/.style={fill = white, circle}]
            \draw[edge] (a) to node {$0$} (b);
            \draw[edge] (a) to node {$0$} (c);
            \draw[edge] (b) to node {$0$} (d);
            \draw[edge] (c) to node {$0$} (b);
            \draw[edge] (c) to node {$0$} (d);
            \draw[edge, bend left] (a) to node {$1$} (z);
            \draw[edge] (b) to node {$1$} (z);
            \draw[edge] (z) to node {$1$} (d);
            \draw[edge, out = 0, in = 0, looseness = 2] (z) to node {$1$} (c);
        \end{scope}
    \end{tikzpicture}
    \captionof*{figure}{Auxiliary TP}
\end{minipage}

\begin{definition}{Auxiliary TP}{}
    Given a digraph $D = (N,A)$ and node demands $b \in R^M$, the \underline{auxiliary TP} $D' = (N', A')$ with node demands $b' \in R^{N'}$ and arc costs $w' \in R^{A'}$ is:
    \begin{itemize}
        \item $N' = N \cup \{z\}$, where $z$ is a new node
        \item $A' = A \cup \{ uz \:\rvert\: u \in N, b(u) < 0\} \cup \{zu \:\rvert\: u \in N, b(u) > 0\}$
        \item $b'(z) = 0, \: b'(v) = b(v) \quad \forall v \in N$
        \item $w'(e) = \begin{cases}
            0 \quad \text{if }e\in A \\
            1 \quad \text{if }e\in A' \setminus A
        \end{cases}$
    \end{itemize}
\end{definition}

\begin{remark}
    ~
    \begin{itemize}
        \item The auxiliary TP has a natural feasible solution (send all flow through $z$, as described above)
        \item The lower bound is $0$ (Every new arc receives $0$ flow).
        \item Combining these 2 observations $\Rightarrow$ An optimal solution exists, and if the optimal solution is $0$, then that means we are sending $0$ flow through the new arcs. So, any flow must be going through the original arcs. This is an initial solution!
    \end{itemize}
\end{remark}

\begin{theorem}{}{}
    A TP is feasible if and only if its auxiliary TP has optimal value $0$
\end{theorem}
And, this behaviour matches that of an auxiliary problem for the two-phase simplex method!

This is great, but still requires us to solve the auxiliary TP in order to determine feasibility of the original. 
Instead, we aim for a characterization that can determine feasibility using only the original TP.
To get there, let's assume that our TP is not feasible, and bootstrap off of what we can determine about our instance through the auxiliary TP.

Assuming the TP is not feasible, the auxiliary TP has an optimal solution with strictly positive optimal value. Let $x^*$ be such a solution that is a also a tree. Let's look at the dual potentials for this solution, setting $y_z = 0$

For any node with an incoming arc $uz$ with non-zero flow in $x^*$, we get that $\bar{w}_{uz} = w_{uz} + y_u - y_z = 0 \Rightarrow 1 + y_u - 0 = 0 \Rightarrow y_u = - 1$.

And, for any node with an outgoing arc $zv$ with non-zero flow in $x^*$, we get that $\bar{w}_{zv} = w_{zv} + y_z - y_v = 0 \Rightarrow 1 + 0 - y_v = 0 \Rightarrow y_v = 1$.

It is easy to check that every other node will have dual potential equal to its parent.

\begin{minipage}{\textwidth}
    \centering
    \begin{tikzpicture}[scale = 0.75]
        \node[vertex][label = {$0$}] (z) at (0,0) {z};
        \node[vertex][label = right:{$1$}] (11) at (-3,-3) {};
        \node[vertex][label = right:{$-1$}] (12) at (0,-3) {};
        \node[vertex][label = right:{$1$}] (13) at (3,-3) {};

        \node[vertex][label = left:{$1$}] (111) at (-4,-6) {};
        \node[vertex][] (112) at (-3,-6) {};
        \node[vertex][label = right:{$1$}] (113) at (-2,-6) {};

        \node[vertex][label = right:{$-1$}] (121) at (0,-6) {};

        \node[vertex][label = right:{$1$}] (131) at (2,-6) {};
        \node[vertex][label = right:{$1$}] (132) at (4,-6) {};

        \draw[densely dotted, thick, shorten < = 3pt] (111) edge (-4,-7); 
        \draw[densely dotted, thick, shorten < = 3pt] (112) edge (-3,-7); 
        \draw[densely dotted, thick, shorten < = 3pt] (113) edge (-2,-7); 
        \draw[densely dotted, thick, shorten < = 3pt] (121) edge (0,-7); 
        \draw[densely dotted, thick, shorten < = 3pt] (131) edge (2,-7); 
        \draw[densely dotted, thick, shorten < = 3pt] (132) edge (4,-7); 

        \draw[edge] (z) edge (11);
        \draw[edge] (12) edge (z);
        \draw[edge] (z) edge (13);

        \draw[edge] (11) edge (111);
        \draw[edge] (11) edge (112);
        \draw[edge] (113) edge (11);
        
        \draw[edge] (12) edge (121);

        \draw[edge] (13) edge (131);
        \draw[edge] (13) edge (132);
    \end{tikzpicture}
\end{minipage}

This labelling partitions the nodes in $N$ into two sets according to their dual potentials.
Let's give these two sets a name: let $S_+$ be the nodes with potential $+1$ and $S_-$ be the nodes with potentials $-1$.
We can reduce our focus to these two sets and $z$, along with the possible arcs that could appear between each entity.

\begin{minipage}{\textwidth}
    \centering
    \begin{tikzpicture}
        \node[vertex][label = {$0$}] (z) at (0,0) {z};
        \node[vertex, minimum size = 2cm][label = above left:{\large $S_-$}] (S-) at (-2,-2) {};
        \node[vertex, minimum size = 2cm][label = above right:{\large $S_+$}] (S+) at (2,-2) {};

        \draw[edge] (z) edge (S+);
        \draw[edge] (S-) edge (z);
        \draw[edge, bend left] (S-) edge (S+);
        \draw[edge, bend left] (S+) edge (S-);
    \end{tikzpicture}
\end{minipage}

In fact, doing a bit more analysis, we find that no arcs go from $S_-$ to $S_+$. Why?
Suppose there is such an arc $e$, then $\overline{w}_e \geq 0$, since otherwise, we would have added such an arc to our tree flow. 
However, doing a bit of calculation, $\overline{w}_e = 0 + (-1) - 1 = -2 < 0$. So, there are no arcs from $S_-$ to $S_+$!

Performing the same analysis on arcs from $S_+$ to $S_-$, we find that the reduced cost:  $\overline{w}_e = 0 + 1 - (-1) = 2 > 0$, and so the flow through this arc must be $0$.
Phew, this is good, since these arcs weren't considered in our tree flow.
So, in fact, our diagram should have one less type of arc.

\begin{minipage}{\textwidth}
    \centering
    \begin{tikzpicture}
        \node[vertex][label = {$0$}] (z) at (0,0) {z};
        \node[vertex, minimum size = 2cm][label = above left:{\large $S_-$}] (S-) at (-2,-2) {};
        \node[vertex, minimum size = 2cm][label = above right:{\large $S_+$}] (S+) at (2,-2) {};

        \draw[edge] (z) edge (S+);
        \draw[edge] (S-) edge (z);
        \draw[edge] (S+) edge (S-);
    \end{tikzpicture}
\end{minipage}

Good, we've done a lot of work now, what can we conclude about the original TP from this?
First, the set $S_-$ pushes a non-zero amount of flow through $z$ towards $S_+$, but receives no flow from $S_+$ (Since all such arcs have $0$-flow). That must mean that $\sum_{v \in S_-} b_v < 0$. 
But, remember that $z$ is a node we added for the auxiliary TP---it doesn't exist in the original TP.
That is, if we remove $z$, then $S_-$ is a set with no outgoing arcs.

In summary, $S_-$ has negative demand, but no outgoing arcs! So, there's nowhere to push the excess flow out from $S_-$. This will be our infeasbility characterization.

\begin{theorem}{}{tp-infeasbility-characterization}
    A TP with digraph $D = (N,A)$ and node demands $b \in \R^N$ is infeasible if and only if there exists $S \subseteq N$ such that $b(S) < 0$ and $\delta(S) = \emptyset$
\end{theorem}
\begin{proof}
    $(\Rightarrow)$ Formalize what we've done above.

    $(\Leftarrow)$ Suppose there exist $S \subseteq N$ with $b(S) < 0$ and $\delta(S) = \emptyset$. We want to show that the TP is infeasible.

    Suppose for a contradiction that there exists a feasible flow $x$. Then
    \begin{align*}
        0 > b(S) &= \sum_{v \in S} b_v \\
        &= \sum_{v \in S} (x(\delta(\overline{v})) - x(\delta(v))) \\
        &= x(\delta(\overline{S})) - x(\delta(S)) \\
        &= x(\delta(\overline{S})) \quad (x(\delta(S)) = 0,\:\text{since}\:\delta(S) = \emptyset) \\
    \end{align*}
    This gives us that $x(\delta(\overline{S})) < 0$, but $x \geq 0$, so $x(\delta(\overline{S})) \geq 0$, giving us our contradiction.
\end{proof}