\subsection{Economic Interpretation}\label{sec:tp_econ-interp}
In this section, we try to provide some intuition for the dual variables by giving an interpretation for node potentials in economic terms.

Imagine transporting a set of commodities through our digraph, and the node potentials represents the prices of these commodities at the nodes.

Say we have the following arc $uv$:

\begin{minipage}{\textwidth}
\centering
\begin{tikzpicture}[scale = 0.75]
    \node[vertex][label={$y_u = 80$}] (u) at (0,0) {u};
    \node[vertex][label={$y_v = 120$}] (v) at (8,0) {v};

    \draw[edge] (u) to node[below, inner sep = 4pt] {$w_{uv} = 70$} (v);
\end{tikzpicture}
\end{minipage}

The reduced cost for this arc is: $\bar{w}_{uv} = 70 + 80 - 120 = 30 > 0$. We can interpret this as: ``Buying one unit of commodity at $u$ for $\$80$, transporting through $uv$ for $\$70$, and selling it at $v$ for $\$120$''. Then, it is clear that we are losing money by using arc $uv$, so we should never increase flow on arcs with positive reduced cost.

Let's look at an example where $\bar{w}_{uv} < 0$

\begin{minipage}{\textwidth}
\centering
\begin{tikzpicture}[scale = 0.75]
    \node[vertex][label={$y_u = 30$}] (u) at (0,0) {u};
    \node[vertex][label={$y_v = 80$}] (v) at (8,0) {v};

    \draw[edge] (u) to node[below, inner sep = 4pt] {$w_{uv} = 40$} (v);
\end{tikzpicture}
\end{minipage}

The reduced costs here is: $\bar{w}_{uv} = 40 + 30 - 80 = -10$. In this case, we buy one unit at $u$ for $\$30$, transport it for $\$40$, and sell it at $v$ for $\$80$. We make $\$10$ by using arc $uv$! So, when the reduced cost is $<0$, we increase flow on that arc.