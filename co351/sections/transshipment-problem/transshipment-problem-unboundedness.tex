\subsection{Unboundedness}
The Network Simplex Algorithm (\Cref{alg:nsm}) we presented previously is not quite complete. 
We have yet to characterize when a TP is unbounded or infeasible.
In this section, we'll focus on characterizing unboundedness.

Unboundedness occurs when no leaving arc can be found, so that you can push flow on the entering arc indefinitely. When does this happen? Say $uv$ is the entering arc and the cycle it creates is $T + uv = C = v_1v_2\ldots v_kv_1$:

\begin{minipage}{\textwidth}
    \centering
    \begin{tikzpicture}[scale = 0.75]
      \node[vertex][label=below left:{$v_k = u$}] (vk) at (0,0) {};
      \node[vertex][label=left:{$v_{k-1}$}] (vk-1) at (-1,3) {};
      \node[vertex][label=above left:{$v_{k-2}$}] (vk-2) at (0,6) {};
      \node[vertex][label=above right:{$v_3$}] (v3) at (6,6) {};
      \node[vertex][label=right:{$v_2$}] (v2) at (7,3) {};
      \node[vertex][label=below right:{$v = v_1$}] (v) at (6,0) {};

      \draw[edge, dashed] (vk) to node[below] {$t$} (v);
      \draw[edge] (v) to node[right] {$+t$} (v2);
      \draw[edge] (v2) to node[right] {$+t$} (v3);
      \draw[edge, dotted] (v3) to node[above] {} (vk-2);
      \draw[edge] (vk-2) to node[left] {$+t$} (vk-1);
      \draw[edge] (vk-1) to node[left] {$+t$} (vk);
    \end{tikzpicture}
\end{minipage}

All arcs are forward arcs, and so we can increase the flow indefinitely.
Let $y_{v_1}$ be the dual potential at $v_1$, then:
\begin{align*}
  y_{v_2} &= y_{v_1} + w_{v_1v_2} \\
  y_{v_3} &= y_{v_2} + w_{v_2v_3} = y_{v_1} + w_{v_1v_2} + w_{v_2v_3} \\
  &\vdots \\
  y_{v_2} &= y_{v_{k-1}} + w_{v_{k-1}v_k} \\
  &= y_{v_1} + w_{v_1v_2} + \ldots + w_{v_{k-1}v_k} \\
\end{align*}
The reduced cost of $uv$ is then:
\begin{align*}
  \bar{w}_{uv} &= y_{v_k} + w_{uv} - y_{v_1} \\
  &=  (y_{v_1} + w_{v_1v_2} + \ldots + w_{v_{k-1}v_k}) w_{uv} - y_{v_1} \\
  &= \sum_{e \in C} w_e < 0 \quad (\text{since } \bar{w}_{uv} < 0)
\end{align*}

\begin{definition}{Negative Dicycle}{}
  A \underline{negative dicycle} is a directed cycle $C$, where the sum of the weights along the cycle is negative, i.e. $\sum_{e \in C} w_e < 0$
\end{definition}

And the presence of a negative dicycle is precisely the characterization of an unbounded TP, since we would be able to send flow through the cycle to reduce cost indefinitely!

\begin{theorem}{}{}
  A feasible TP is unbounded if and only if there exists a negative dicycle
\end{theorem}
\begin{proof}
$(\Leftarrow)$ Let $C$ be a negative dicycle with $w(C) < 0$ and $x^*$ be a feasible flow.
Define a flow $x^C$, where $x^C_e = \begin{cases} 
t \quad \text{if } e \in C \\ 
0 \quad \text{if } e \not\in C
\end{cases}$.
So, $x^C(\delta(\bar{v})) - x^C(\delta(v)) = 0$ for each $v \in N$ since $C$ is a dicycle.

Then, $x^* + x^C$ is a feasible flow, since:
\begin{align*}
  &(x^* + x^C)(\delta(\bar{v})) - (x^* + x^C)(\delta(v)) \\ 
  &=x*(\delta(\bar{v})) - x*(\delta(v)) + x^C(\delta(\bar{v})) - x^C(\delta(v)) \\ 
  & = b_v + 0 = b_v
\end{align*}
And the objective value using $x^* + x^C$ is:
\begin{align*}
  w^\intercal(x^* + x^C) &= w^\intercal x^* + w^\intercal x^C \\
  &= w^\intercal x^* + t \cdot w(C)
\end{align*}
While $w^\intercal x^*$ is fixed, we can make $t \cdot w(C)$ arbitarily large, and as $t \longrightarrow \infty$, $t \cdot w(C) \longrightarrow -\infty$ (since $w(C) < 0$), so the total objective value $w^\intercal(x^* + x^C) \longrightarrow \infty$

$(\Rightarrow)$ TODO
\end{proof}

