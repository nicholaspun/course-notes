\makeatletter
\def\input@path{{../}}
\makeatother
\documentclass[../co351.tex]{subfiles}

\begin{document}
\section{Graph Theory Primer}
Let $G = (V, E)$ be a graph, where $V$ is the vertex set and $E$ is the edge set.
\begin{definition}{}{}
   The \textbf{degree} of a vertex $v \in V$ (denoted $\deg(v)$) is the number of edges with one end in $v$. (i.e. The size of the set $\{ va \:\rvert\: a \in V, \ a \neq v \}$).

   A \textbf{walk} is a sequence of vertices $v_1v_2\ldots v_k$ where $v_iv_{i+1}$ is an edge. A \textbf{path} is a walk where all vertices are distinct. A \textbf{cycle} is a walk where $v_1 = v_k$ and $v_1, \ldots v_{k-1}$ are distinct.

   Finally, we say a graph is \textbf{connected} if there exists a path between any two vertices in $G$.
\end{definition}

\begin{definition}{}{}
  For $S \subset V$, the \textbf{cut} induced by $S$ is the set of all edges with one end in $S$ and one end not in $S$, denoted $\delta(S) = \{uv \in E \:\rvert\: u \in S, \ v \not\in S \}$. Given two vertices $s, t \in V$ with $s \in S, t \not\in S$, we call $\delta(S)$ an \textbf{$s,t$-cut}

  An \textbf{$s,t$-path} is a path with starting vertex $s$ and ends on $t$.
\end{definition}

\begin{theorem}{}{}
  There exists an $s,t$-path if and only if every $s,t$-cut is nonempty
\end{theorem}

\begin{definition}{}{}
  A \textbf{tree} is a connected graph with no cycles. A \textbf{spanning tree} is a subgraph that is a tree and has vertex set $V$
\end{definition}
Note the following:
\begin{itemize}
  \item A tree on $n$ vertices contains $n-1$ edges.
  \item If $T$ is a tree, then adding an edge $uv \not\in T$ creates exactly one cycle $C$. Moreover, if $xy$ is an edge in $C$, then $T + uv - xy$
\end{itemize}

Let $D = (N, A)$ be a directed graph. $N$ is a set of nodes and $A$ is a set of ordered pairs of nodes (called arcs).

\begin{definition}{}{}
  For an arc $(u,v)$, we call $u$ the \textbf{tail} and $v$ the \textbf{head}.

  The \textbf{out-degree} of node $u$ (denoted $d(u)$ or $d^{\text{out}}(u)$) is the number of arcs with tail $u$. The \textbf{in-degree} of node $u$ (denoted $d(\overline{u})$ or $d^{\text{in}}(u)$) is the number of arcs with head $u$.

  A \textbf{diwalk} is a sequence of nodes $v_1v_2\ldots v_k$ where $(v_i, v_{i+1})$ is an arc. \textbf{Dipaths} and \textbf{Dicycle} are defined analgous to simple graphs but with arcs instead of edges.

  For $S \subset N$, the \textbf{cut} induced by $S$ is denoted $\delta(S) = \{ (u,v) \in A \:\rvert\: u \in S, v \not\in S \}$. (sometimes written as $\delta^{\text{out}}(S)$) This is the set of arcs with tail in $S$.
  We denote the complement of $S$ by $\overline{S}$, and define $\delta(\overline{S}) = \{ (u,v) \in A \:\rvert\: u \not\in S, v \in S \}$ (sometimes written as $\delta^{\text{in}}(S))$) to be the set of arcs with head in $S$. Finally, if $s \in S, t \not\in S$, then $\delta(S)$ is an $s,t$-cut.
\end{definition}

\begin{theorem}{}{}
  There is an $s,t$-dipath if and only if every $s,t$ cut is non-empty
\end{theorem}
\begin{proof}
($\Rightarrow$) Suppose there exists an empty $s,t$-cut $\delta(S)$. This partitions the graph into two sets of nodes $S$ and $N\setminus S$, with $s \in S$ and $t \in N\setminus S$ and no outgoing edges from $S$ to $N\setminus S$. As such, an $s,t$-dipath cannot exist.

($\Leftarrow$) Suppose every $s,t$-cut is non-empty and let $S$ be the set of nodes $v \in A$ where a $s,v$-dipath exists. If $t \in S$, then we're done, so suppose $t \not\in S$. Then, $\delta(S)$ is an $s,t$-cut.
By assumption, $\delta(S)$ is non-empty and so there is an arc $(x,y) \in \delta(S)$ with $x \in S$ and $y \in N\setminus S$.
Since $x \in S$, an $s,x$-dipath $P$ exists and since $y \not\in S$, there does not exist an $s,y$-dipath, but $P + (x,y)$ is one. This is a contradiction
\end{proof}

\begin{definition}{}{}
  The \textbf{node-arc incidence matrix} $M$ of a digraph $D = (N, A)$ is a matrix of $|N|$ rows and $|A|$ columns, such that:
  \begin{itemize}
    \item The rows correspond to the nodes of $D$,
    \item The columns correspond to the arcs of $D$,
    \item And the entry for node $u$ and arc $(i,j)$, denoted $m_{u,ij}$ is given by:
    \[
    m_{u,ij} =
    \begin{cases}
      0 & \text{if }u \neq i \text{ and }u \neq j, \\
      +1 & \text{if }u = j \text{ and}, \\
      -1 & \text{if }u = i \\
    \end{cases}
    \]
  \end{itemize}
\end{definition}

\begin{example}{}{}
An example of a digraph $D = (N, A)$ (on the left) and its corresponding node-arc incidence matrix on the right.

\begin{minipage}[t]{\textwidth}
  \begin{minipage}[c]{0.4\textwidth}
    \centering
    \begin{tikzpicture}[node distance = 2cm]
      \node[vertex] (1) {1};
      \node[vertex] (2) [right of=1] {2};
      \node[vertex] (3) [below of=1] {3};
      \node[vertex] (4) [below right of=3, below of=2, anchor=5] {4};
      \node[vertex] (5) [below left of=3] {5};

      \draw[edge] (1) edge (2);
      \draw[edge] (1) edge (4);
      \draw[edge] (2) edge (4);
      \draw[edge] (3) edge (1);
      \draw[edge] (3) edge (5);
      \draw[edge] (4) edge (3);
      \draw[edge] (4) edge (5);
      \draw[edge] (5) edge (1);
    \end{tikzpicture}
  \end{minipage}
  \hfill
  \begin{minipage}[c]{0.6\textwidth}
    \centering
    $\begin{blockarray}{rrrrrrrrr}
    & 12 & 14 & 24 & 31 & 35 & 43 & 45 & 51 \\
    \begin{block}{l[rrrrrrrr]}
      1 & -1 & -1 &  0 &  1 &  0 &  0 &  0 &  1 \\
      2 &  1 &  0 & -1 &  0 &  0 &  0 &  0 &  0 \\
      3 &  0 &  0 &  0 & -1 & -1 &  1 &  0 &  0 \\
      4 &  0 &  1 &  1 &  0 &  0 & -1 & -1 &  0 \\
      5 &  0 &  0 &  0 &  0 &  1 &  0 &  1 & -1 \\
    \end{block}
  \end{blockarray}
  $
  \end{minipage}
\end{minipage}
\end{example}

\end{document}
