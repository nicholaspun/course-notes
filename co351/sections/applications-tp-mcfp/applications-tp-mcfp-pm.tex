\subsection{Minimum Bipartite Perfect Matching (MBPM)}

\underline{\textbf{Goal:}} Given a bipartite graph $G = (V = A \cup B, E)$, with $|A| = |B|$, and edge costs $w \in \R^E$, find a perfect matching in $G$ of minimum total cost.

\begin{example}{Perfect Matchings}{applications-tp-mcfp_pm}
    Below is a complete bipartite graph $G$ with bipartiion $\{1,2,3\}$ and $\{a,b,c\}$ and 2 perfect matchings of $G$

    \begin{minipage}{\textwidth}
        \centering
        \begin{tikzpicture}
            \node[vertex] (2) at (0,0) {2};
            \node[vertex] (3) at (4,0) {3};
            \node[vertex] (1) at (-4, 0) {1};

            \node[vertex] (b) at (0,-4) {b};
            \node[vertex] (c) at (4,-4) {c};
            \node[vertex] (a) at (-4,-4) {a};
            
            \begin{scope}[every node/.style={fill = white, ellipse}]
                \draw (1) to node {\small $5$} (a);
                \draw (1) to node[pos = 0.8] {\small $10$} (b);
                \draw (1) to node[pos = 0.2] {\small $4$} (c);

                \draw (2) to node[pos = 0.8] {\small $3$} (a);
                \draw (2) to node[pos = 0.2] {\small $14$} (b);
                \draw (2) to node[pos = 0.8] {\small $15$} (c);

                \draw (3) to node[pos = 0.2] {\small $9$} (a);
                \draw (3) to node[pos = 0.8] {\small $2$} (b);
                \draw (3) to node {\small $6$} (c);
            \end{scope}
        \end{tikzpicture}
        \captionof*{figure}{Complete Bipartite Graph $G$}
    \end{minipage}

    \begin{minipage}{\textwidth}
        \begin{minipage}{0.5\textwidth}            
            \centering
            \begin{tikzpicture}[scale = 0.5]
                \node[vertex] (2) at (0,0) {2};
                \node[vertex] (3) at (4,0) {3};
                \node[vertex] (1) at (-4, 0) {1};

                \node[vertex] (b) at (0,-4) {b};
                \node[vertex] (c) at (4,-4) {c};
                \node[vertex] (a) at (-4,-4) {a};
                
                \begin{scope}[every node/.style={fill = white, ellipse}]
                    \draw (1) to node {\small $5$} (a);    
                    \draw (2) to node {\small $14$} (b);
                    \draw (3) to node {\small $6$} (c);
                \end{scope}
            \end{tikzpicture}
            \captionof*{figure}{Perfect matching with cost $25$}
        \end{minipage}
        \begin{minipage}{0.5\textwidth}            
            \centering
            \begin{tikzpicture}[scale = 0.5]
                \node[vertex] (2) at (0,0) {2};
                \node[vertex] (3) at (4,0) {3};
                \node[vertex] (1) at (-4, 0) {1};
    
                \node[vertex] (b) at (0,-4) {b};
                \node[vertex] (c) at (4,-4) {c};
                \node[vertex] (a) at (-4,-4) {a};
                
                \begin{scope}[every node/.style={fill = white, ellipse}]
                    \draw (1) to node {\small $4$} (c);
                    \draw (2) to node[pos = 0.75] {\small $3$} (a);
                    \draw (3) to node[pos = 0.25] {\small $2$} (b);
                \end{scope}
            \end{tikzpicture}
            \captionof*{figure}{Perfect matching with cost $9$}
        \end{minipage}
    \end{minipage}
\end{example}

We will formulate this as an MCFP.

Direct each edge from $A$ to $B$. 
For each arc, set the capacity to be $1$ And, for each node, set $b_v = \begin{cases} -1 \: \text{if }v \in A \\ 1 \: \text{if } v \in B \end{cases} $. 
The arc costs stay the same.
For example, using the graph from \Cref{exmp:applications-tp-mcfp_pm}, our resulting digraph for the MCFP instance becomes:

\begin{minipage}{\textwidth}
    \centering
    \begin{tikzpicture}
        \node[vertex] (2) at (0,0) {2};
        \node[vertex] (3) at (4,0) {3};
        \node[vertex] (1) at (-4, 0) {1};

        \node[vertex] (b) at (0,-4) {b};
        \node[vertex] (c) at (4,-4) {c};
        \node[vertex] (a) at (-4,-4) {a};
        
        \begin{scope}[every node/.style={fill = white, ellipse}]
            \draw[edge] (1) to node {\footnotesize $(5,1)$} (a);
            \draw[edge] (1) to node[pos = 0.8] {\footnotesize $(10,1)$} (b);
            \draw[edge] (1) to node[pos = 0.2] {\footnotesize $(4,1)$} (c);

            \draw[edge] (2) to node[pos = 0.8] {\footnotesize $(3,1)$} (a);
            \draw[edge] (2) to node[pos = 0.2] {\footnotesize $(14,1)$} (b);
            \draw[edge] (2) to node[pos = 0.8] {\footnotesize $(15,1)$} (c);

            \draw[edge] (3) to node[pos = 0.2] {\footnotesize $(9,1)$} (a);
            \draw[edge] (3) to node[pos = 0.8] {\footnotesize $(2,1)$} (b);
            \draw[edge] (3) to node {\footnotesize $(6,1)$} (c);
        \end{scope}
    \end{tikzpicture}
    \captionof*{figure}{MCFP Formulation for \Cref{exmp:applications-tp-mcfp_pm} \\
    The arcs are labelled $(x_e, c_e)$}
\end{minipage}

We want to show a correspondence between optimal solutions to this MCFP and MBPM. 
\begin{theorem}{Integrality Condition}{}
    If an MCFP has an optimal solution, and all capacities and node demands are integers, then there exists an integral optimal solution
\end{theorem}
\begin{proof}
    The proof is based on the NSM algorithm. 
    Since node demands are integral, we can always start off with an integral basic feasible solution.
    Then, in every iteration, we modify our feasible solution by pushing/removing flow until an arc reaches capacity or $0$ flow.
    But capacities are also integral, so in each iteration where we modify the feasible solution, we remain integral.
\end{proof}
By this theorem, we can say that for our MCFP formulation, there exists an integer-valued optimal solution $x$.
So, $x_e = 0$ or $x_e = 1$ for each arc $e$ and $M = \{e\:|\:x_e = 1\}$ is exactly a perfect matching because each node in $A$ can push out at most $1$ unit of flow, and so is incident with exactly one edge in $M$.

In the reverse direction, we also note that any PM corresponds to an integral flow. So, our MCFP formulation solves the MBPM problem.