\section{Evolution of Graphs and Theorem E}

\begin{theorem}[Theorem E]\label{th:E}
    ~
    \begin{enumerate}[label = (\alph*)]
        \item Fix $k \geq 2$ integer.
        If $n^{\frac{k-2}{k-1} - 2} \ll p \ll n^{\frac{k-1}{k} - 2}$, then a.a.s $\G(n, p)$ is a forest and the largest component is of order $k$.

        \item If $p \ll \frac{1}{n}$, then a.a.s $\G(n, p)$ is a forest and the largest component is of order $o(\log n)$
        
        \item If $p = \frac{c}{n}$, $0 < c < 1$, then a.a.s every component of $\G(n, p)$ is a tree or unicyclic and the largest component has order $\Theta(\log n)$
        
        \item If $p = \frac{c}{n}$, $c > 1$, then a.a.s $\G(n, p)$ contains a unique component of linear order and all other components of order $\bigO(\log n)$
        
        \item When $p \geq \frac{\log n + \log\log n \omega(1)}{2n}$, a.a.s $\G(n, p)$ has a giant component and a few isolated vertices
        
        \item When $p \geq \frac{\log n + \omega(1)}{n}$, a.a.s $\G(n, p)$ connected and has a perfect matching if $n$ even, or a matching of size $\frac{n-1}{2}$ if $n$ is odd.
        
        \item When $p \geq \frac{\log n + \log\log n + \omega(1)}{n}$, a.a.s $\G(n, p)$ is Hamiltonian
    \end{enumerate}
\end{theorem}

\subsection{Small Subgraphs}
\begin{lemma}
    If $p = o(1/n)$, then a.a.s $\G(n, p)$ has no cycles
\end{lemma}
\begin{proof}
    Use Markov's Inequality
\end{proof}

\begin{proof}
    (of \Cref{th:E} (a))

    That $\G(n, p)$ is a forest is directly implied by above.
    It remains to show that every tree has order $\leq k$ and there is one tree of order $k$.

    Let $X_t$ be the number of trees of order $t$ in $\G(n, p)$.
    First, we show that that $\E \left(\sum_{t \geq k + 1} X_t\right) = o(1)$, so a.a.s $\sum_{t \geq k + 1}X_t = 0$ by Markov's inequality.
    This tells us that we don't have trees of order $> k$

    Then, we show the existence of a tree of order $k$ by using the 2nd moment method.
\end{proof}