% !TeX root = ../cs487.tex

\section{Multiplication Time, Newton Iteration and Fast Division with Remainder}

\subsection{Multiplication Time}
\begin{definition}{}{}
    A function $M:\N_{> 0} \rightarrow \R_{>0}$ is a \ul{multiplication time} for $R[x]$ ($R$ ring) if polynomials in $R[x]$ of degree $< n$ can be multiplied using at most $M(n)$ ring operations in $R$.
\end{definition}
We'll use $M$ to add information to our cost estimates and improve our cost analysis for algorithms

\begin{example}{}{}
    We've seen a couple examples of multiplication time already:
    \begin{itemize}
        \item For \textbf{Na\"{i}ve Multiplication} of polynomials, we know that $M(n) \in \bigO(n^2)$
        \item Using \textbf{Karatsuba's algorithm}, we can reduce that cost to $M(n) \in \bigO(n^{1.59})$
        \item Cantor \& Kaltofen (\Cref{th:lec6-ring-poly-mult}) showed: $M(n) \in \bigO(n^2)$
    \end{itemize}
    And a couple interesting results for integers:
    \begin{itemize}
        \item Sch\"{o}hage \& Strassen (\Cref{th:lec6-fast-int-mult}) showed that: $M(n) \in \bigO(n\log n\log \log n)$
        \item F\"{u}rer showed in 2007: $M(n) \in \bigO(n\log n K^{\log^* n})$, where $K$ is some constant $> 1$ and $\log^*$ is the iterated logarithm. (Harvey and Van Der Hoeven showed that $K = 4$ in 2018)
        \item In March 2019, Harvey and Van Der Hoeven showed that $M(n) \in \bigO(n \log n)$ (Note that the result has yet to be officially peer-reviewed as of the time these notes were taken)
    \end{itemize}
\end{example}

\ul{Useful Assumptions about $M$}:
\begin{enumerate}
    \item Superlinearity: 
    
    If $n \geq m$ 
    \begin{equation}\label{eq:lec7-mult-time-superlinearity}
        \frac{M(n)}{n} \geq \frac{M(m)}{m}
    \end{equation}
    
    \item At most quadratic:
    \begin{equation}\label{eq:lec7-mult-time-quadratic}
        M(mn) \leq m^2M(n)
    \end{equation}
\end{enumerate}

\begin{proposition}{}{lec7-mult-time}
    \Cref{eq:lec7-mult-time-superlinearity} implies the following:
    \begin{itemize}
        \item $M(mn) \geq mM(n)$
        \item $M(n + m) \geq M(n) + M(m)$
        \item $M(n) \geq n$
    \end{itemize}
\end{proposition}

\begin{example}{}{}
    Using the assumptions and \Cref{prop:lec7-mult-time}, we can say the following:
    \begin{itemize}
        \item $nM(n) + M(n^2) \leq 2M(n^2) \in \bigO(M(n^2))$
        \item $M(cn) \leq c^2M(n) \in \bigO(M(n))$ for constant $c$ 
        \item $n^3 + nM(n) \geq M(n^3) + M(n^{\frac{3}{2}}) \in \Omega(M(n^3))$
    \end{itemize}
\end{example}

\subsection{Fast Division with Remainder}
\ul{Goal}: Given two polynomials: 
\begin{align*}
    a(x) &= a_nx^n + a_{n-1}x^{n-1} + \ldots + a_1x + a_0 \in \F[x] \\
    b(x) &= b_mx^m + b_{m-1}x^{m-1} + \ldots + b_1x + b_0 \in \F[x].
\end{align*} 
with $a_n, b_m \neq 0$, $b$ monic, and $m \leq n$, find $q(x)$ and $r(x)$ such that $a(x) = q(x)b(x) + r(x)$, $\deg r < \deg b$.

\subsubsection{Reversions}
To solve the above problem, we'll need a new operation called \ul{reversion}. 

Given $a(x) = a_nx^n + a_{n-1}x^{n-1} + \ldots + a_1x + a_0 \in \F[x]$, we define the reversion of $a$, denoted $\rev(a)$ or $\rev_n(a)$ to be the following procedure:
\begin{enumerate}
    \item Substitute $x$ with $\frac{1}{y}$, then
    \item Multiply by $y^n$.
\end{enumerate} 
This gives:
\begin{equation}
    \begin{aligned}
        \rev(a) = \rev_n(a) &:= y^na\left(\frac{1}{y}\right) \\
        &= y^n\left(a_0 + a_1\left(\frac{1}{y}\right) + \ldots + a_n\left(\frac{1}{y^n}\right)\right) \\
        &= y^na_0 + y^{n-1}a_1 + \ldots + a_n
    \end{aligned}
\end{equation}
So, we have reversed the ordering of the coefficients.

\begin{remark}
    $\rev\left(\rev\left(a\right)\right) = a$
\end{remark}

\begin{note}
    Reversions don't cost any ring operations. For example, if the coefficients were stored as an array, then a reversion is just a reversal of the array. 
\end{note}

\subsubsection{Back to Fast Division}
Using the idea of reversions, let's rewrite our goal.
The reversion of $a(x) = q(x)b(x) + r(x)$ is:
\begin{align*}
    \rev_n(a) = y^na\left(\frac{1}{y}\right) &= y^n\left(q\left(\frac{1}{y}\right)b\left(\frac{1}{y}\right) + r\left(\frac{1}{y}\right)\right) \\
    &= y^n\left(q\left(\frac{1}{y}\right)\right)
    y^n\left(b\left(\frac{1}{y}\right)\right) + 
    y^n\left(r\left(\frac{1}{y}\right)\right) \\
    &= \rev_{n-m}(q)\rev_m(b) + y^{n-m+1}\rev_{m-1}(r)
\end{align*}
For the last step: $\deg b = m$, so $\deg q = n - m$.
Further, since $\deg r < \deg b$, $\deg r$ is at most $m-1$.

And, if we take the equation modulo $y^{n - m + 1}$, this gives:
\begin{equation}\label{eq:lec7-fast-division-new-goal}
    \rev_n(a) = \rev_{n-m}(q)\rev_m(b) \pmod{y^{n-m+1}}
\end{equation}
\ul{We revise our goal}: Solve \Cref{eq:lec7-fast-division-new-goal} for the unknown $\rev_{n-m}(q)$

