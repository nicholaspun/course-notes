% !TeX root = ../cs487.tex

\section{Polynomial Multiplication using Lagrange Interpolation, Vandermonde Matrix}

\subsection{Polynomial Multiplication using Lagrange Interpolation}
We continue our discussion on polynomial multiplication.
Again, the motivation for the following algorithm is to reduce the \ul{non-scalar cost}.

\begin{theorem}{}{}
    Given $a,b \in \F[x]$, $\deg a, \deg b < n$, multiplying $a \times b$ has cost $2n - 1$ non-scalar multiplications if $\#\F \geq 2n - 1$
\end{theorem}
\begin{note}
    The non-scalar multiplications refer to the coefficients of the polynomials we want to multiply
\end{note}

\begin{idea}
Use Polynomial Evaluation and Interpolation.
\end{idea}
Let's see an example of this:

\begin{example}{}{}
    We want to multiply the following polynomials $a(x) = 2 + 3x, b(x) = 1 + 2x$ using Lagrange Interpolation.

    Let $c(x) = a(x) \times b(x)$ be the resulting polynomial.
    Note that $\deg c = 2$ so we'll need 3 evaluation points.
    Let $u_0 = 0, u_1 = 1, u_2 = 2$ be the 3 such points. 

    Evaluating our polynomials $a$ and $b$ at these 3 points give:
    \begin{align*}
        &a(u_0) = 2, \ b(u_0) = 1 \\
        &a(u_1) = 5, \ b(u_1) = 3 \\
        &a(u_2) = 8, \ b(u_2) = 5 
    \end{align*}
    which gives us the value of $c$ at the 3 points:
    \begin{align*}
        c(u_0) &= a(u_0) \times b(u_0) = 2 \\
        c(u_1) &= a(u_1) \times b(u_1) = 15 \\
        c(u_2) &= a(u_2) \times b(u_2) = 40 \\
    \end{align*}
    We're nearly there! 
    Now we have 3 data points:$(0, 2), (1, 15), (2, 40)$.
    Define the following Lagrange basis polynomials:
    \begin{equation*}
        L_0 =  \frac{(x-1)(x-2)}{(0-1)(0-2)}, \
        L_1 =  \frac{(x-0)(x-2)}{(1-0)(1-2)}, \
        L_2 =  \frac{(x-0)(x-1)}{(2-0)(2-1)} \
    \end{equation*}
    Then,
    \begin{equation*}
        c(x) = 2 \times L_0 + 15 \times L_1 + 40 \times L_2 = 2 + 7x + 6x^2
    \end{equation*}
\end{example}

\begin{proof}
Consider the following algorithm:

\begin{algorithm}[H]
    \caption{Polynomial Multiplication using Lagrange Interpolation}\label{alg:poly-mult-lagrange}

    (Our input is: $a,b \in \F[x]$ with $\deg a, \deg b < n$)

    \BlankLine
    \nl Choose $2n - 1$ evaluation points: $u_1, \ldots u_{2n - 1} \in \F$

    \nl Compute $\alpha_i = a(u_i)$ and $\beta_i = b(u_i)$ for $i = 1,\ldots, 2n - 1$

    \nl \label{step:non-scalar-mults} Compute $\gamma_i = \alpha_i\beta_i$ for $i = 1,\ldots, 2n - 1$ 

    \nl Interpolate to get $c = a \times b$ using Lagrange's formula:
    \begin{equation}
        c = \sum_{1 \leq i \leq 2n-1} \gamma_i L_i
    \end{equation}
    where $L_i$ is defined as:
    \begin{equation}
        L_i = \prod_{j \neq i} \frac{x - u_j}{u_i - u_j} \in \F[x]
    \end{equation}
\end{algorithm}
Only \Cref{step:non-scalar-mults} contributes to the non-scalar cost, and only $2n - 1$ multiplications are made.
\end{proof}

\subsection{A Slight Detour: The Vandermonde Matrix}
\begin{definition}{Vandermonde Matrix}{}
    We define the \ul{Vandermonde Matrix} to be the following $n \times n$ matrix:
    \begin{equation}
        VDM(u_1, u_2, \ldots, u_n) = 
        \begin{bmatrix}
            1 & u_1^1 & \ldots & u_1^{n-1} \\
            1 & u_2^1 & \ldots & u_2^{n-1} \\
            \vdots & \vdots & \ddots & \vdots \\
            1 & u_n^1 & \ldots & u_n^{n-1} \\
        \end{bmatrix}
    \end{equation}
\end{definition}

Now, given $a(x) = a_0 + a_1x + \ldots a_{n-1}x^{n-1}$, observe that we can express both polynomial evaluation and interpolation using the Vandermonde matrix.

\subsubsection{Polynomial Evaluation}
Given $a$ as above and $n$ evaluation points: $u_0, \ldots u_{n - 1}$, the evaluation of $a$ at these $n$ points is:
\begin{equation}\label{eq:vandermonde-poly-eval}
    VDM(u_0, \ldots, u_{n-1})
    \begin{bmatrix}
        a_0 \\ \vdots \\ a_{n-1}
    \end{bmatrix}
    =
    \begin{bmatrix}
        a(u_0) \\ \vdots \\ a(u_{n-1})
    \end{bmatrix}
\end{equation}

\subsubsection{Polynomial Interpolation}
\begin{proposition}{Determinant of a Vandermonde Matrix}{det-vandermonde}
    Let $V = VDM(u_1, \ldots, u_n)$. The determinant $\det(V)$ is:
    \begin{equation}\label{eq:vandermonde-poly-interp}
        \det(V) = \prod_{1 \leq i < j \leq n} (u_j - u_i)
    \end{equation}
\end{proposition}
\begin{remark}
    Observe that when $u_1, \ldots, u_n$ are all distinct, then $\det(V)$ is non-zero and the matrix is invertible.
\end{remark}
Given $(u_0, \alpha_0), (u_1, \alpha_1), \ldots, (u_{n-1}, \alpha_{n-1})$, where $u_0, \ldots u_{n-1}$ are all distinct, the interpolation of $a$ at these $n$ points is:
\begin{equation}
    \begin{bmatrix}
        a_0 \\ \vdots \\ a_{n-1}
    \end{bmatrix}
    =
    VDM(u_0, \ldots, u_{n-1})^{-1}
    \begin{bmatrix}
        \alpha_0 \\ \vdots \\ \alpha_{n-1}
    \end{bmatrix}
\end{equation}
(The inverse exists by \Cref{prop:det-vandermonde})

\subsection{Another View on Polynomial Multiplication}
We can rewrite the steps \Cref{alg:poly-mult-lagrange} as expressions involving the Vandermonde matrix.
Here, we'll just look at an example, and leave writing the algorithm out formally as an exercise to the reader.

\begin{example}{}{}
    We will work over the field $\Z_7$.
    Consider the following polynomials:
    \begin{align*}
        f(x) &= 2x^2 + 3x + 1 \\
        g(x) &= x^2 + 5x + 2
    \end{align*}

    Let $h(x) = f(x) \times g(x) = h_0 + h_1x + \ldots + h_4x^4$ and let's choose the evaluation points: 0, 1, 2, 3, 4 (We'll see very soon that we can choose better points)
    \begin{enumerate}
        \item \ul{Evaluation}:

        To evaluate $f, g$ at the 4 evaluation points, we'll use \Cref{eq:vandermonde-poly-eval}:
        \begin{equation*}
            VDM(0,1,2,3,4)
            \begin{blockarray}{cc}
                f & g \\
                \begin{block}{[c|c]}
                    \bigstrut[t] 1 & 2 \\
                    3 & 5 \\
                    2 & 1 \\
                    0 & 0 \\
                    0 & 0 \bigstrut[b] \\
                \end{block}
            \end{blockarray}
            =
            \begin{blockarray}{cc}
                f & g \\
                \begin{block}{[c|c]}
                    \bigstrut[t] 1 & 2 \\
                    6 & 1 \\
                    1 & 2 \\
                    4 & 2 \\
                    4 & 4 \bigstrut[b] \\
                \end{block}
            \end{blockarray}
        \end{equation*} 
        (Note that the $0$'s are used for padding since we want to evaluate $5$ points but our polynomials are only of length $3$)

        \item \ul{Pointwise Multiplication}:

        Now we take the resulting evaluated points and perform pointwise multiplication to obtain $h(x)$.
        That is:
        \begin{align*}
            h(0) = f(0) \times g(0) &= 1 \times 2 = 2 \\
            h(1) = f(1) \times g(1) &= 6 \times 1 = 6 \\ 
            h(2) = f(2) \times g(2) &= 1 \times 2 = 2 \\
            h(3) = f(3) \times g(3) &= 4 \times 2 = 1 \\
            h(4) = f(4) \times g(4) &= 4 \times 4 = 2
        \end{align*}

        \item \ul{Interpolation}:
        
        Finally, use \Cref{eq:vandermonde-poly-interp} to find $h_0, \ldots, h_4$:
        \begin{equation*}
            VDM(0,1,2,3,4)^{-1}
            \begin{bmatrix}
                2 \\ 6 \\ 2 \\ 1 \\ 2
            \end{bmatrix}
            = 
            \begin{bmatrix}
                2 \\ 4 \\ 6 \\ 6 \\ 2
            \end{bmatrix}
            = 
            \begin{bmatrix}
                h_0 \\ h_1 \\ h_2 \\ h_3 \\ h_4
            \end{bmatrix}
        \end{equation*}
    \end{enumerate}

    So, $h(x) = 2x^4 + 6x^3 + 6x^2 + 4x + 2$
\end{example}

\subsection{Choosing ``Good'' Evaluation Points}
\begin{definition}{Primitive $n$-th root of unity}{}
    Let $n \in \N$ and $w \in \F$. $w$ is a \ul{primitive $n$-th root of unity} ($n$-PRU) if:
    \begin{enumerate}
        \item $w^n = 1$
        \item $n$ is a unit in $\F$
        \item $w^k \neq 1$ for $1 \leq k < n$
    \end{enumerate}
\end{definition}
\begin{remark}
    For the 2nd property, we mean the $n$-fold sum of the additive identity $1_{\F}$ in $\F$.
    Further, we can define primitive $n$-th roots of unity arbitrary rings as well.
    In this case, the requirement that $n$ is a unit is more significant.
    We will see later that the inverse of such $n$ must exist for our particular usage of these elements.
\end{remark}

\begin{example}{PRUs}{}
    \begin{enumerate}
        \item Let $\F = \C$:
        \begin{itemize}
            \item $w = e^{\frac{2\pi i}{8}}$ is an $8$-PRU.
            \item $w = -1$ is an $4$-PRU
            \item $w = i$ is an $2$-PRU
        \end{itemize}

        \item Let $\F = \Z_{17}$:
        \begin{itemize}
            \item $w = 3$ is an $16$-PRU
            \item $w = 7$ is an $4$-PRU
        \end{itemize}
    \end{enumerate}
\end{example}

\begin{proposition}{}{}
    \begin{enumerate}
        \item If $w$ is an $n$-PRU, then $w^{-1}$ is also an $n$-PRU
        \item If $w$ is an $n$-PRU and $n$ is even, then $w^2$ is an $\frac{n}{2}$-PRU
    \end{enumerate}
\end{proposition}

