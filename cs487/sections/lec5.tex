% !TeX root = ../cs487.tex

\section{Lagrange Interpolants for Polynomial Multiplications}
We continue our discussion on polynomial multiplication.
Again, the motivation for the following algorithm is to reduce the \underline{non-scalar cost}.

\subsection{Using Lagrange Interpolants}
\begin{theorem}{}{}
    Given $a,b \in \F[x]$, $\deg a, \deg b < n$, multiplying $a \times b$ has cost $2n - 1$ non-scalar multiplications if $\#\F \geq 2n - 1$
\end{theorem}
\begin{note}
    The non-scalar multiplications refer to the coefficients of the polynomials we want to multiply
\end{note}

\begin{idea}
Use Polynomial Evaluation and Interpolation.
\end{idea}
Let's see an example of this:

\begin{example}{}{}
    We want to multiply the following polynomials $a(x) = 2 + 3x, b(x) = 1 + 2x$ using Lagrange Interpolation.

    Let $c(x) = a(x) \times b(x)$ be the resulting polynomial
    First, we note that $\deg a \times b = 2$ so we'll need 3 evaluation points.
    Let $u_0 = 0, u_1 = 1, u_2 = 2$ be our evaluation points.
    Evaluating our polynomials $a$ and $b$ at these 3 points give:
    \begin{align*}
        &a(u_0) = 2, \ b(u_0) = 1 \ \Rightarrow \ a(u_0)b(u_0) = 2 \\
        &a(u_1) = 5, \ b(u_1) = 3 \ \Rightarrow \ a(u_1)b(u_1) = 15 \\
        &a(u_2) = 8, \ b(u_2) = 5 \ \Rightarrow \ a(u_2)b(u_2) = 40 \\
    \end{align*}

    Then, using the Lagrange interpolants:  
    % \begin{equation*}
        
    % \end{equation*}
\end{example}