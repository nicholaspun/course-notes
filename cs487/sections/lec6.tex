% !TeX root = ../cs487.tex

\section{Discrete Fourier Transform}

\begin{definition}{}{}
    Let $w$ be an $n$-PRU in $\F$.
    Define $V(w)$ to be the following $n$-by-$n$ matrix:
    \begin{equation}
        V(w) = 
        \begin{bmatrix}
            1 & 1 & \ldots & 1 \\    
            1 & w^1 & \ldots & w^{n-1} \\    
            1 & w^2 & \ldots & w^{2(n-1)} \\    
            \vdots & \vdots & \ddots & \vdots \\    
            1 & w^{(n-1)} & \ldots & w^{(n-1)^2} \\    
        \end{bmatrix}
        =
        VDM(w^0, w^1, \ldots, w^{n-1})
    \end{equation}
    and likewise for $V(w^{-1})$.
\end{definition}

\begin{theorem}{}{}
    Let $w$ be an $n$-PRU, then $V(w)\cdot V(w^{-1}) = nI$
\end{theorem}
\begin{proof}
    \begin{align*}
        \left(V(w)V(w^{-1})\right) &= i\text{-th row of }V(w) \times j\text{-th col of }V(w^{-1}) \\
        &= \sum_{0 \leq k < n} w^{ik}w^{-jk} \\ 
        &= \sum_{0 \leq k < n} w^{(i - j)k}
    \end{align*}
    If $i = j$, then the sum is $\sum_k 1 = n$

    If $i \neq j$, then this is a geometric series:
    \begin{equation*}
        \sum_{0 \leq k < n} w^{(i - j)k} = \frac{w^{(i-j)n} - 1}{w^{(i - j)} - 1} = 0
    \end{equation*}
    since $w^{(i - j)n} = 1$ as $w$ is an $n$-PRU
\end{proof}

\begin{definition}{Discrete Fourier Transform}{}
    Let $w \in \F$ be an $n$-PRU. $\DFT(w)$ is the linear map $F^n \rightarrow F^n$ defined by:
    \begin{equation}
        \begin{bmatrix}
            a_0 \\ a_1 \\ \vdots \\ a_{n-1}
        \end{bmatrix}
        \longmapsto
        \begin{bmatrix}
            b_0 \\ b_1 \\ \vdots \\ b_{n-1}
        \end{bmatrix}
        =
        V(w)
        \begin{bmatrix}
            a_0 \\ a_1 \\ \vdots \\ a_{n-1}
        \end{bmatrix}
    \end{equation}
    i.e. $b_j = \sum_{0 \leq k < n} a_kw^{jk}$
\end{definition}

Our goal is to develop a fast way to evaluate this.
Let's look at a motivating example:

Let $f = a_0 + a_1x + \ldots + a_kx^k$ and consider evaluating $f(1)$ and $f(-1)$.
$1, -1$ weren't picked at random, observe that $(-1)^2 = 1^2$, and likewise for any even power.
This suggests that we can save some computations on even exponents.

Decompose $f$ like so:
\begin{equation*}
    f(x) = 
\end{equation*}