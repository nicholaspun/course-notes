% !TeX root = ../cs487.tex

\section{Discrete Fourier Transform}

\begin{definition}{}{}
    Let $w$ be an $n$-PRU in $\F$.
    Define $V(w)$ to be the following $n$-by-$n$ matrix:
    \begin{equation}
        V(w) = 
        \begin{bmatrix}
            1 & 1 & \ldots & 1 \\    
            1 & w^1 & \ldots & w^{n-1} \\    
            1 & w^2 & \ldots & w^{2(n-1)} \\    
            \vdots & \vdots & \ddots & \vdots \\    
            1 & w^{(n-1)} & \ldots & w^{(n-1)^2} \\    
        \end{bmatrix}
        =
        VDM(w^0, w^1, \ldots, w^{n-1})
    \end{equation}
    and likewise for $V(w^{-1})$.
\end{definition}

\begin{theorem}{}{}
    Let $w$ be an $n$-PRU, then $V(w)\cdot V(w^{-1}) = nI$
\end{theorem}
\begin{proof}
    \begin{align*}
        \left(V(w)V(w^{-1})\right) &= i\text{-th row of }V(w) \times j\text{-th col of }V(w^{-1}) \\
        &= \sum_{0 \leq k < n} w^{ik}w^{-jk} \\ 
        &= \sum_{0 \leq k < n} w^{(i - j)k}
    \end{align*}
    If $i = j$, then the sum is $\sum_k 1 = n$

    If $i \neq j$, then this is a geometric series:
    \begin{equation*}
        \sum_{0 \leq k < n} w^{(i - j)k} = \frac{w^{(i-j)n} - 1}{w^{(i - j)} - 1} = 0
    \end{equation*}
    since $w^{(i - j)n} = 1$ as $w$ is an $n$-PRU
\end{proof}

\begin{definition}{Discrete Fourier Transform}{}
    Let $w \in \F$ be an $n$-PRU. $\DFT(w)$ is the linear map $F^n \rightarrow F^n$ defined by:
    \begin{equation}
        \begin{bmatrix}
            a_0 \\ a_1 \\ \vdots \\ a_{n-1}
        \end{bmatrix}
        \longmapsto
        \begin{bmatrix}
            b_0 \\ b_1 \\ \vdots \\ b_{n-1}
        \end{bmatrix}
        =
        V(w)
        \begin{bmatrix}
            a_0 \\ a_1 \\ \vdots \\ a_{n-1}
        \end{bmatrix}
    \end{equation}
    i.e. $b_j = \sum_{0 \leq k < n} a_kw^{jk}$
\end{definition}

Our goal is to develop a fast way to evaluate this.
Let's look at a motivating example:

Let $f = a_0 + a_1x + \ldots + a_kx^k$ (with $k$ even) and consider evaluating $f(1)$ and $f(-1)$.
Observe that $(-1)^2 = 1^2$ ($1, -1$ weren't picked at random!), and in fact, this holds for an even power.
This suggests that we can save some computations on even exponents.

Decompose $f$ like so:
\begin{align*}
    f(x) &= (a_0 + a_2x^2 + a_4x^4 + \ldots) + x(a_1 + a_3x^2 + a_5x^4 + \ldots) \\
    &= \sum_{0 \leq i \leq k/2} a_{2i}x^{2i} + x\sum_{0 \leq j \leq k/2} a_{2j + 1}x^{2j}
\end{align*}
Now, we only have even exponents in the sums!
Define the following two functions:
\begin{align}
    f_{even}(x) &= \sum_{0 \leq i \leq k/2} a_{2i}x^{i} \\
    f_{odd}(x) &= \sum_{0 \leq j \leq k/2} a_{2j + 1}x^{j}
\end{align}
Then, $f$ can be rewritten using these two functions like so:
\begin{equation}\label{eq:lec6-function-decompose}
    f(x) = f_{even}(x^2) + xf_{odd}(x^2)
\end{equation}

What happens if we try using \Cref{eq:lec6-function-decompose} to evaluate $f$ at $1, -1$?
Let's try it:

\begin{minipage}{0.4\textwidth}
    \centering
    \begin{align*}
        f(1) &= f_{even}(1^2) + (1)f_{odd}(1^2) \\
        &= f_{even}(1) + (1)f_{odd}(1)
    \end{align*}
\end{minipage}
\begin{minipage}{0.6\textwidth}
    \centering
    \begin{align*}
        f(-1) &= f_{even}((-1)^2) + (-1)f_{odd}((-1)^2) \\
        &= f_{even}(1) + (-1)f_{odd}(1)
    \end{align*}
\end{minipage}

~\newline So, the evaluation of $f_{even}(1)$ and $f_{odd}(1)$ can be reused between the two computations!
Further observe that $f_{even}$ and $f_{odd}$ are of half the degree of $f$.
We've reduced the evaluation of $f$ at the two points ($1$ and $-1$), to evaluating two polynomials, of half the degree, at a single point (just $1$).

Let's look at another example using the $4$ points: $1, i, - 1, - i$.
Plugging these $4$ values into equation \Cref{eq:lec6-function-decompose}, we get the following expressions:
\begin{align*}
    f(1) &= f_{even}(1) + (1)f_{odd}(1) \\
    f(i) &= f_{even}(i^2) + (i)f_{odd}(1^2) \\
    f(1) &= f_{even}(1) + (-1)f_{odd}(1) \\
    f(1) &= f_{even}(i^2) + (-i)f_{odd}(i^2)
\end{align*}
which suggests that we can limit our evaluations to only the points $1$ and $i$. (So we pair up the points $(1, -1), (i, -i)$)

In general, it seems that we can use \Cref{eq:lec6-function-decompose} to save evaluations if we have $n$ points of the form:
\begin{equation*}
    (u_1, -u_1), (u_2, -u_2), \ldots, (u_{\frac{n}{2}}, -u_{\frac{n}{2}})
\end{equation*}

\begin{theorem}{}{}
    Let $n$ be a power of $2$. Let $w \in \F$ be an $n$-PRU. Then, $\DFT(w)$ can be computed in $\bigO(n\log n)$ field operations.
\end{theorem}

\begin{lemma}{}{}
    
\end{lemma}

\textcolor{red}{TODO: Finish this lecture ... }

\begin{definition}{}{}
    We say $\F$ \ul{supports} the FFT (\textcolor{red}{TODO: Add Cref here}), if $\F$ has a $2^\ell$-PRU for any $\ell \in \N$
\end{definition}

\begin{theorem}{}{}
    If $\F$ supports the FFT, the polynomials of degree at most $n$ can be multiplied ... \textcolor{red}{wut this theorem makes no sense ... hmmm i'll fix this later}
\end{theorem}

\begin{theorem}{Sch\"{o}nhage \& Strassen, 1971}{}
    Integer Multiplication can be done in time $\bigO(n\log n \log\log n)$ 
\end{theorem}

\begin{theorem}{Cantor \& Kaltofen, 1991}{}
    Over any ring, polynomials of degree $n$ can be multiplied in $\bigO(n \log n \log\log n)$
\end{theorem}
