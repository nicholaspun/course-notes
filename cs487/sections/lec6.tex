% !TeX root = ../cs487.tex

\section{Discrete Fourier Transform and Fast Fourier Transform}

\subsection{Discrete Fourier Transform}

\begin{definition}{}{}
    Let $w$ be an $n$-PRU in $\F$.
    Define $V(w)$ to be the following $n$-by-$n$ matrix:
    \begin{equation}
        V(w) = 
        \begin{bmatrix}
            1 & 1 & \ldots & 1 \\    
            1 & w^1 & \ldots & w^{n-1} \\    
            1 & w^2 & \ldots & w^{2(n-1)} \\    
            \vdots & \vdots & \ddots & \vdots \\    
            1 & w^{(n-1)} & \ldots & w^{(n-1)^2} \\    
        \end{bmatrix}
        =
        VDM(w^0, w^1, \ldots, w^{n-1})
    \end{equation}
    and likewise for $V(w^{-1})$.
\end{definition}

\begin{theorem}{}{}
    Let $w$ be an $n$-PRU, then $V(w)\cdot V(w^{-1}) = nI$
\end{theorem}
\begin{proof}
    \begin{align*}
        \left(V(w)V(w^{-1})\right) &= i\text{-th row of }V(w) \times j\text{-th col of }V(w^{-1}) \\
        &= \sum_{0 \leq k < n} w^{ik}w^{-jk} \\ 
        &= \sum_{0 \leq k < n} w^{(i - j)k}
    \end{align*}
    If $i = j$, then the sum is $\sum_k 1 = n$

    If $i \neq j$, then this is a geometric series:
    \begin{equation*}
        \sum_{0 \leq k < n} w^{(i - j)k} = \frac{w^{(i-j)n} - 1}{w^{(i - j)} - 1} = 0
    \end{equation*}
    since $w^{(i - j)n} = 1$ as $w$ is an $n$-PRU
\end{proof}

\begin{definition}{Discrete Fourier Transform}{}
    Let $w \in \F$ be an $n$-PRU. $\DFT(w)$ is the linear map $F^n \rightarrow F^n$ defined by:
    \begin{equation}
        \begin{bmatrix}
            a_0 \\ a_1 \\ \vdots \\ a_{n-1}
        \end{bmatrix}
        \longmapsto
        \begin{bmatrix}
            b_0 \\ b_1 \\ \vdots \\ b_{n-1}
        \end{bmatrix}
        =
        V(w)
        \begin{bmatrix}
            a_0 \\ a_1 \\ \vdots \\ a_{n-1}
        \end{bmatrix}
    \end{equation}
    i.e. $b_j = \sum_{0 \leq k < n} a_kw^{jk}$
\end{definition}
\begin{note}
    We may write $\DFT(w)(f)$, where $f$ is a function with $\deg f = n - 1$, to denote performing the DFT on the coefficients of $f$
\end{note}

\subsection{Fast Fourier Transform}
Our goal is to develop a fast algorithm to evaluate $\DFT(w)(f)$.
The main idea is to divide-and-conquer.
Let's look at a motivating example:

Let $f = a_0 + a_1x + \ldots + a_kx^k$ (with $k$ even).
Consider the decomposition of $f$ like so:
\begin{align*}
    f(x) &= (a_0 + a_2x^2 + a_4x^4 + \ldots) + (a_1x + a_3x^3 + a_5x^5 + \ldots) \\
    &= (a_0 + a_2x^2 + a_4x^4 + \ldots) + x(a_1 + a_3x^2 + a_5x^4 + \ldots) \\
    &= \sum_{0 \leq i \leq k/2} a_{2i}x^{2i} + x\sum_{0 \leq j \leq k/2} a_{2j + 1}x^{2j}
\end{align*}

That is, we divide the function into parts containing only even or only odd exponents.
Then, we factor an $x$ out of the odd exponents so that we only have even exponents in the sums.
Define the following two functions:
\begin{align}
    f_{even}(x) &= \sum_{0 \leq i \leq k/2} a_{2i}x^{i} \\
    f_{odd}(x) &= \sum_{0 \leq j \leq k/2} a_{2j + 1}x^{j}
\end{align}

Then, $f$ can be rewritten using these two functions like so:
\begin{equation}\label{eq:lec6-function-decompose}
    f(x) = f_{even}(x^2) + xf_{odd}(x^2)
\end{equation}

Now consider evaluating $f$ at the 4 points: $1, i, -1, -i$.
Plugging these $4$ values into equation \Cref{eq:lec6-function-decompose}, we get the following expressions:
\begin{align*}
    f(1) &= f_{even}(1) + (1)f_{odd}(1) \\
    f(i) &= f_{even}(i^2) + (i)f_{odd}(i^2) \\
    f(1) &= f_{even}(1) + (-1)f_{odd}(1) \\
    f(1) &= f_{even}(i^2) + (-i)f_{odd}(i^2)
\end{align*}
Evaluating $1$ and $-1$ amounts to computing $f_{even}(1)$ and $f_{odd}(1)$ and the combining the results appropriately.
Likewise, we can do the same with $i$ and $-i$

This suggests that we can reuse the computations of $f_{even}$ and $f_{odd}$, which are polynomials of half the degree of $f$.

In general, if we can ``pair up" our $n$ points in the form:
\begin{equation*}
    (u_1, -u_1), (u_2, -u_2), \ldots, (u_{\frac{n}{2}}, -u_{\frac{n}{2}})
\end{equation*}
we can use \Cref{eq:lec6-function-decompose} to save evaluations

\begin{lemma}{}{}
    Let $w$ be an $n$-PRU. 
    Then, there is always a pairing of points of the above form.
    More precisely:
    \begin{equation*}
        w^{\frac{n}{2} + i} = -w^i
    \end{equation*}
    for $i = 1, \ldots, \frac{n}{2}$
\end{lemma}
\begin{proof}
    $w^n = 1 \Rightarrow w^n - 1 = 0 \Rightarrow \left(w^{\frac{n}{2}} - 1 \right)\left(w^{\frac{n}{2}} + 1\right) = 0$.
    So, $w^{\frac{n}{2}} = \pm 1$, but $w^{\frac{n}{2}} \neq 1$ since $w$ is an $n$-PRU, so $w^{\frac{n}{2}} = -1$.
    Then, $w^{\frac{n}{2} + i} = w^{\frac{n}{2}}w^i = -w^i$
\end{proof}

\begin{theorem}{}{lec6-fft}
    Let $n$ be a power of $2$. Let $w \in \F$ be an $n$-PRU. Then, $\DFT(w)$ can be computed in $\bigO(n\log n)$ field operations.
\end{theorem}
\begin{proof}
Let $f = a_0 + a_1x + \ldots + a_{n-1}x^{n-1}$.
We'll exhibit an algorithm to compute $\DFT(w)(f)$ in $\bigO(n\log n)$ time:

\begin{algorithm}[H]
    \caption{Fast Fourier Transform (FFT)}\label{lec6:alg-fft}

    \ul{Input}: $f = a_0 + a_1x + \ldots + a_{n-1}x^{n-1}$, $w \in \F$ an $n$-PRU.
    
    \ul{Output}: $\DFT(w)(f)$
    
    \BlankLine
    \nl Compute $w^2, w^3, \ldots, w^{n-1}$

    \nl Recursively compute $\DFT(w)(f_{even})$ and $\DFT(w)(f_{odd})$:
    \begin{equation*}
        \DFT(w)(f_{even}) = 
        \begin{pmatrix}
            f_{even}(w^2) \\ f_{even}(w^4) \\ \vdots \\ f_{even}(w^{n})
        \end{pmatrix}
        \quad\text{and}\quad
        \DFT(w)(f_{odd}) =
        \begin{pmatrix}
            f_{odd}(w^2) \\ f_{odd}(w^4) \\ \vdots \\ f_{odd}(w^{n})
        \end{pmatrix}
    \end{equation*}

    \nl Compute $f(w^k) = f_{even}(w^{2k}) + w^kf_{odd}(w^{2k})$ for $k = 0, 1, \ldots, n - 1$
\end{algorithm}

Let $T(n)$ be the cost for $\deg f = n$. 
Line 1 costs less than $n$ multiplications.
Line 3 costs $n$ multiplications and $n$ additions.
In total,the cost of \Cref{lec6:alg-fft} is
\begin{equation*}
    T(n) = 2T\left(\frac{n}{2}\right) + 3n \in \bigO(n\log n)
\end{equation*} 
\end{proof}

\subsection{Polynomial Multiplication using the FFT}
We want to relate this method back to polynomial multiplication.

\begin{theorem}{}{}
    Let $\F$ be a field, $n = 2^k$, $w \in \F$ an $n$-PRU.
    Then, polynomials in $\F[x]$ of degree $< \frac{n}{2}$ can be multiplied using $\bigO(n \log n)$ field ops.
\end{theorem}
\begin{proof}
    Recall that polynomial multiplication can be performed by:
    \begin{enumerate}
        \item Evaluating the two functions at $n$ points
        \item Multiplying the images pointwise
        \item Interpolating to obtain the multiplied function
    \end{enumerate}

    Further, we can express evaluation and interpolation as matrix operations using the Vandermonde matrix (\Cref{exmp:lec5:poly-mult-vdm}).
    Now, we'll use the DFT and Fast Fourier Transform algorithm to speed both operations up.

    Let $a = a_0 + a_1x + \ldots + a_{\frac{n}{2} - 1}x^{\frac{n}{2} - 1}$, $b = b_0 + b_1x + \ldots + b_{\frac{n}{2} - 1}x^{\frac{n}{2} - 1}$ and
    \begin{equation*}
        \overline{a} = 
        \begin{bmatrix}
            a_0 \\ a_1 \\ \vdots \\ a_{\frac{n}{2} - 1} \\ 0 \\ \vdots \\ 0    
        \end{bmatrix}
        \qquad
        \overline{b} = 
        \begin{bmatrix}
            b_0 \\ b_1 \\ \vdots \\ b_{\frac{n}{2} - 1} \\ 0 \\ \vdots \\ 0    
        \end{bmatrix}
    \end{equation*}
    where the $0$'s are padding elements so that the vectors are of length $n$.

    Let $c = c_0 + c_1 + \ldots + c_nx^n = a \times b$ and $\overline{c}$ be the vector of its coefficients as with $\overline{a}$ and $\overline{b}$.
    Then,
    \begin{align*}
        \overline{c} &= \left(\DFT(w)\right)^{-1}\left(\DFT(w)(a) \cdot \DFT(w)(b)\right) \\
        &= \frac{1}{n}\left(\DFT(w^{-1})\right)\left(\DFT(w)(a) \cdot \DFT(w)(b)\right)
    \end{align*}
    (where $\cdot$ is pointwise multiplication)

    And this uses $\bigO(n\log n)$ field operations
\end{proof}

\begin{definition}{}{lec6-supports}
    We say $\F$ \ul{supports} the FFT (\Cref{lec6:alg-fft}), if $\F$ has a $2^\ell$-PRU for any $\ell \in \N$
\end{definition}

More generally, we'll use \Cref{defn:lec6-supports} to extend \Cref{th:lec6-fft}.
\begin{theorem}{}{}
    If $\F$ supports the FFT, the polynomials of degree at most $n$ can be multiplied in $\bigO(n \log n)$ field ops.
\end{theorem}

\begin{theorem}{Sch\"{o}nhage \& Strassen, 1971}{lec6-fast-int-mult}
    Integer Multiplication can be done in time $\bigO(n\log n \log\log n)$ 
\end{theorem}

\begin{theorem}{Cantor \& Kaltofen, 1991}{lec6-ring-poly-mult}
    Over any ring, polynomials of degree $n$ can be multiplied in $\bigO(n \log n \log\log n)$
\end{theorem}
