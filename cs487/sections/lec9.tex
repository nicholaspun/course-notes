% !TeX root = ../cs487.tex

\section{Chinese Remainder Algorithm}

\begin{theorem}{Chinese Remainder Theorem}{chinese-remainder-theorem}
    Let $R$ be an Euclidean Domain and $M = m_0m_1 \ldots m_{r-1}$, where $\gcd(m_i, m_j) = 1$ for $i \neq j$.
    (Note that this condition implies that $M = \lcm(m_0, \ldots, m_1)$)
    Then,
    \begin{equation}
        \faktor{R}{(M)} \cong \faktor{R}{(m_0)} \times \ldots \times \faktor{R}{(m_{r-1})}
    \end{equation}
\end{theorem}

Recall that in \Cref{sec:lec2-multimodular-reduction},  that showed one direction of the bijection, i.e. the map taking $\rem(a, M) \mapsto \left(\rem(a, m_0), \rem(a, m_1), \ldots, \rem(a, m_{r-1})\right)$.
This takes $\bigO(\log^2 M)$ word ops.

In this lecture, we want to develop the reverse map using a similar amount of word ops.

\subsection{Why is this interesting?}
