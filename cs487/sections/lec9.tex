% !TeX root = ../cs487.tex

\section{Chinese Remainder Algorithm}

\begin{theorem}{Chinese Remainder Theorem}{chinese-remainder-theorem}
    Let $R$ be an Euclidean Domain and $M = m_0m_1 \ldots m_{r-1}$, where $\gcd(m_i, m_j) = 1$ for $i \neq j$.
    (Note that this condition implies that $M = \lcm(m_0, \ldots, m_1)$)
    Then,
    \begin{equation}
        \faktor{R}{(M)} \cong \faktor{R}{(m_0)} \times \ldots \times \faktor{R}{(m_{r-1})}
    \end{equation}
\end{theorem}

Recall that in \Cref{sec:lec2-multimodular-reduction},  that showed one direction of the bijection, i.e. the map taking $\rem(a, M) \mapsto \left(\rem(a, m_0), \rem(a, m_1), \ldots, \rem(a, m_{r-1})\right)$.
This takes $\bigO(\log^2 M)$ word ops.

In this lecture, we want to develop the reverse map using a similar amount of word ops.
The idea of using small moduli and this Chinese Remainder algorithm (which we may refer to as ``Chinese Remaindering'') to obtain a pre-image will be a central idea in many of the remaining algorithms we will learn.

\begin{example}{}{}
    Operations on modular residues are reflected when we perform Chinese Remainder to retrieve the preimage.

    Let $m = 1001 = 7 \times 11 \times 13$, so in this example we'll consider $R = \Z_{1001} \cong \Z_7 \times \Z_{11} \times \Z_{13}$
    
    Let $a \equiv 233 \pmod{m}, \ b \equiv 365 \pmod{m}$.
    By \Cref{th:chinese-remainder-theorem}, $a \mapsto (2, 2, 12)$ and $b \mapsto (1, 2, 1)$
    
    Consider the following operations:
    \begin{enumerate}
        \item $\rem(a + b, m) = (2, 2, 12) + (1, 2, 1) = (3, 4, 0) \mapsto 598 \pmod{m}$.
        Indeed: $233 + 365 \equiv 598 \pmod{m}$!

        \item $\rem(a\cdot b, m) = (2, 2, 12)\cdot(1, 2, 1) = (2, 4 , 12) \mapsto 961 \pmod{m}$.
        Indeed, we can check: $233 \cdot 365 = 85045 \equiv 961 \pmod{m}$ 
    \end{enumerate}

    \bf{The main takeaway}: The Chinese Remainder Theorem guarantees a unique preimage, so we can always work with small moduli and then work backwards.
    (Many times, it'll be much more computationally efficient to do so!)
\end{example}

Let's start developing the algorithm:

\ul{Goal}: Given moduli $m_0, m_1, \ldots m_{r - 1}$ and images $v_0, v_1, \ldots, v_{r - 1}$.
Find an $f \in R$ ($R$ Euclidean Domain) such that $f \equiv v_i \pmod{m_i}$ for $0 \leq i \leq r - 1$

Let $M = m_0m_1\ldots m_{r-1}$ and consider the following construction for $f$:
\begin{equation}
    f = v_0s_0\frac{M}{m_0} + v_1s_1\frac{M}{m_1} + \ldots + v_{r-1}s_{r-1}\frac{M}{m_{r-1}}
\end{equation}
Consider $f \bmod m_0$, then each of $\frac{M}{m_1}, \ldots, \frac{M}{m_{r-1}}$ will vanish and we are left with:
\begin{align*}
    f &= v_0s_0\frac{M}{m_0} \equiv v_0 \pmod{m_0} \\
    &\Rightarrow s_0\frac{M}{m_0} \equiv 1 \pmod{m_0}
\end{align*}
To determine $s_0$ recall that we can apply EEA to get:
\begin{equation*}
    s_0\frac{M}{m_0} + t_0m_0 = 1
\end{equation*}
Since the choice of $s_0$ was arbitrary, this works for any $s_i$ for $0 \leq i \leq r - 1$.
So our algorithm is:

\begin{algorithm}[H]
    \caption{Chinese Remainder Algorithm}\label{chinese-remainder-algorithm}

    \nl Compute $M = m_0m_1\ldots m_{r-1}$

    \nl Compute $\frac{M}{m_i}$ for $0 \leq i \leq r - 1$

    \nl Compute $s_i$ such that $s_i\frac{M}{m_i} + t_im_i = 1$ using EEA

    \nl Compute $f = v_0s_0\frac{M}{m_0} + v_1s_1\frac{M}{m_1} + \ldots + v_{r-1}s_{r-1}\frac{M}{m_{r-1}}$
\end{algorithm}

\begin{theorem}{}{}
    If each $v_i$ is in the range $[0, m_i - 1]$, then the cost of the Chinese Remainder algorithm is $\bigO(\log^2 M)$ word operations
\end{theorem}
\begin{proof}
    Each step is bounded by $\bigO(\log^2 M)$ word operations
\end{proof}

\begin{example}{}{}
    Let $m_0, m_1, m_2 = 7, 11, 13$ and $v_0, v_1, v_2 = 2, 2, 12$, so $M = 1001$

    We want to find $s_0, s_1, s_2$ so that $f \equiv v_i \pmod{m_i}$ for each $i$, where:
    \begin{align*}
        f &= 2 \times \frac{7 \times 11 \times 13}{7} \times s_0 + 2 \times \frac{7 \times 11 \times 13}{11} \times s_1 + 12 \times \frac{7 \times 11 \times 13}{13} \times s_2 \\
        &= 2 \times (11 \times 13) \times s_0 + 2 \times (7 \times 13) \times s_1 + 12 \times (7 \times 11) \times s_2 \\
    \end{align*}

    We use EEA to compute each $s_i$:
    \begin{align*}
        &\gcd(11 \times 13, 7) = 1 \Rightarrow (-2)(11 \times 13) + (41)(7) = 1 \Rightarrow s_0 = -2 \\
        &\gcd(7 \times 13, 11) = 1 \Rightarrow (4)(7 \times 13) + (-33)(11) = 1 \Rightarrow s_1 = 4 \\
        &\gcd(7 \times 11, 13) = 1 \Rightarrow (-1)(7 \times 11) + (6)(13) = 1 \Rightarrow s_2 = -1    
    \end{align*}
\end{example}

\begin{note}
    Here, we perform the algorithm over the positive range, i.e. All our numbers are in the range $[0, m - 1]$.
    We can perform the Chinese Remainder algorithm over the symmetric range as well.
    Then, our numbers must be in the range: $\left[-\left\lfloor\frac{m-1}{2}\right\rfloor, \left\lfloor\frac{m}{2}\right\rfloor\right]$
\end{note}

\subsection{Variations to Chinese Remaindering}

\subsubsection{Incremental Chinese Remaindering}
Suppose we are given some $a \in \Z$ and we start choosing small primes $m_0, m_1, m_2, \ldots$ to compute the sequence:
\begin{equation*}
    \rem(a, m_0), \rem(a, m_0m_1), \rem(a, m_0m_1m_2), \ldots
\end{equation*}
Eventually, $M = m_0m_1m_2\ldots$ will become large enough such that $\rem(a, M)$ will be the actual result.
This occurs when the sequence fixed (There may be a chance that the sequence appears fixed but $M$ isn't large enough yet, however, we can perform the analysis to determine that with low probability, we will get false positives).

\subsubsection{Mixed Radix Representation}
Let $0 \leq a < m_0m_1\ldots m_{r-1}$, each $m_i \in \N_{\geq 2}$ (not necessarily relatively prime)

\ul{Claim}: We can write $a$ uniquely as:
\begin{equation*}
    a = a_0 + a_1m_0 + a_2m_0m_1 + \ldots + a_rm_0m_1\ldots m_{r-1}
\end{equation*}
with $0 \leq a_i < m_i$ for all $i$.

In fact, we can use this to help us compute the incremental chinese remaindering by letting $m_0m_1\ldots m_{r-2}$ be one radix and $m_r$ be the second radix.